\chapter{Implementasi, Pengujian, dan Eksperimen}
\label{chap:Implementasi, Pengujian, dan Eksperimen}

\section{Deskripsi Perangkat Keras dan Lunak yang Digunakan}
\label{sec:desc_perangkat}
Pembangunan perangkat lunak pada skripsi ini dilakukan dengan menggunakan aplikasi perangkat lunak \textit{IntelliJ IDEA Ultimate} 2017.1. Semua kode ditulis dengan menggunakan bahasa pemrograman \textit{Java} dan bahasa pemrograman berbasis web (HTML, CSS, Javascript). \textit{Framework Hadoop} yang digunakan adalah Hadoop versi 2.7.2. Hadoop dipasang pada 5 komputer yang ada di Laboratorium Komputasi, Jurusan Teknik Informatika, Fakultas Teknologi Informasi dan Sains, Universitas Katolik Parahyangan. Berikut merupakan spesifikasi komputer - komputer yang digunakan:
\begin{itemize}
	\item Sistem Operasi: Ubuntu 14.04 LTS
	\item Tipe SO: 64-bit
	\item Prosesor: Intel® Core™ i3 CPU 550 @ 3.20GHz × 4 
	\item Memori: 7,7 GiB
	\item Grafik: Gallium 0.4 on AMD REDWOOD
	\item Disk: 264,2 GB
\end{itemize}
Sebuah komputer dijadikan \textit{master} dan sisanya dijadikan sebagai \textit{slave}.

\section{Implementasi Antarmuka}
\label{sec:impl_antarmuka}

Antarmuka yang dibangun pada keseluruhan perangkat lunak ada 2 jenis. (1) Antarmuka perangkat lunak jenis pertama diperuntukkan untuk modul yang berbasis \textit{MapReduce}, dirancang dengan antarmuka \textit{shell}. (2) Antarmuka perangkat lunak jenis kedua diperuntukkan untuk modul yang tidak berbasiskan \textit{MapReduce}, seperti modul kelola input dan klasifikasi. Antarmuka pada jenis kedua ini dirancang menggunakan antarmuka berbasis web HTML.

\subsection{\textit{Shell}}
Perangkat lunak pada modul berbasis \textit{MapReduce} yang sudah di-\textit{compile} dengan java dalam bentuk jar dijalankan melalui \textit{shell} pada \textit{terminal} dengan perintah:
\begin{lstlisting}
:~$ hadoop jar MapReduce.jar /bayes/<model-directory>
\end{lstlisting}
Perintah ini menerima 1 buah parameter yang menentukan model dari direktori yang digunakan untuk proses \textit{MapReduce} yang dijalankan. Parameter tersebut memiliki \textit{namespace} yang perlu selalu diikutsertakan untuk menentukan model direktori pada saat menjalankan program, yaitu namespace \texttt{/bayes/}. Setelah \texttt{/bayes/}, kata berikutnya merupakan nama model direktori yang digunakan untuk menjalankan program. Misalnya, menjalankan program pada model bernama \texttt{car}, maka perintah yang perlu dijalankan adalah:
\begin{lstlisting}
:~$ hadoop jar MapReduce.jar /bayes/car
\end{lstlisting}
Berikut ini merupakan contoh yang salah:
\begin{lstlisting}
:~$ hadoop jar MapReduce.jar /car
\end{lstlisting}

Untuk mengingatkan penggunaan program, perlu diingat kembali urutan yang perlu dilakukan untuk menjalankan keseluruhan perangkat lunak dengan benar pada \textit{modul specification} Gambar~\ref{fig:Modul Specification}. 

Di dalam model direktori yang dijadikan parameter saat menjalankan program menggunakan \textit{shell}, sudah harus memiliki beberapa direktori yang diproses pada modul kelola input, direktori serta isinya. Berikut merupakan contoh isi dalam model direktori minimal yang dibutuhkan untuk menjalankan perangkat lunak berbasis \textit{MapReduce} dengan menjalankan perintah untuk melihat seluruh isi direktori dan file secara rekursif:
\begin{lstlisting}
:~$ hadoop fs -ls -R /bayes/car
drwxrwxr-x   - hduser supergroup          0 2017-04-19 22:17 /bayes/car/info
-rw-r--r--   3 hduser supergroup        117 2017-04-19 22:17 /bayes/car/info/meta.info
drwxrwxr-x   - hduser supergroup          0 2017-04-19 22:17 /bayes/car/input
-rw-r--r--   3 hduser supergroup  569163610 2017-04-19 22:29 /bayes/car/input/dataset-1_19-04-2017_10-17.in
drwxrwxr-x   - hduser supergroup          0 2017-04-19 22:17 /bayes/car/testing
drwxrwxr-x   - hduser supergroup          0 2017-04-19 22:17 /bayes/car/testing/input
-rw-r--r--   3 hduser supergroup  232561301 2017-04-19 22:29 /bayes/car/testing/input/dataset-1_19-04-2017_10-17.in
\end{lstlisting}

\subsection{Berbasis Web HTML}
Antarmuka pada modul klasifikasi dan modul kelola input dibuat dengan menggunakan antarmuka berbasis web yaitu HTML. Seperti yang telah dijelaskan pada bagian~\ref{sec:Perancangan Antarmuka}, pada modul yang tidak berbasiskan \textit{MapReduce} ini, memiliki 6 buah antarmuka dan 1 layout untuk menu yang disimpan disamping kiri dari keseluruhan tampilan antarmuka.

\subsubsection{Layout Menu}
Gambar~\ref{fig:Implementasi Antarmuka Layout Menu} menunjukkan antarmuka untuk tampilan layout menu yang ada pada bagian kiri seluruh tampilan antarmuka yang dibuat. Antarmuka ini dibuat untuk memudahkan user dalam melakukan perpindahan halaman antarmuka sesuai yang diinginkan oleh user.
\begin{figure}[H]
	\centering
	\includegraphics[scale=0.65]{ImplGUI/layout_menu}
	\caption[Implementasi Antarmuka Layout Menu]{Implementasi Antarmuka Layout Menu}
	\label{fig:Implementasi Antarmuka Layout Menu}
\end{figure}


\subsubsection{Dashboard}
Gambar~\ref{fig:Implementasi Antarmuka Dashboard} menunjukkan antarmuka tampilan awal pada modul \textit{non-MapReduce}. Seperti yang sudah dijelaskan di bagian rancangan antarmuka \textit{Dashboard} pada bagian~\ref{subsec:Dashboard}, pada tampilan ini user bisa melakukan monitor model NBC yang telah dimasukkan ke dalam perangkat lunak ini, seperti melihat seluruh atribut kelas dan atribut prediktor, lalu frekuensi kemunculan untuk setiap atribut kelas dan atribut prediktor bertipe diskrit, dan juga nilai rata - rata dan standard deviasi untuk atribut prediktor bertipe numerik. 

Terdapat keterangan dari nama kolom pada bagian \textit{Bayesian Model} yang perlu diperhatikan. \verb|P_N = Predictor Name|; \verb|P_V = Predictor Value|; \verb|C_N = Class Name|; \verb|C_V = Class Value|.

\begin{figure}[H]
	\centering
	\includegraphics[scale=0.65]{ImplGUI/dashboard}
	\caption[Implementasi Antarmuka Dashboard]{Implementasi Antarmuka Dashboard}
	\label{fig:Implementasi Antarmuka Dashboard}
\end{figure}


\subsubsection{Input Set Manager}
Gambar~\ref{fig:Implementasi Antarmuka Input Set Manager} menunjukkan antarmuka untuk melakukan pengelolaan file input untuk ditransfer ke dalam HDFS yang nantinya menjadi file untuk data latih dan data testing algoritma klasifikasi \textit{naive bayes}. Seperti yang sudah dijelaskan di bagian rancangan antarmuka \textit{Input Set Manager} pada~\ref{subsec:Input Set Manager}, bahwa user dapat memilih model direktori; user dapat memilih file input dan file info; user dapat memilih presentase penggunaan file input untuk dijadikan data \textit{training} dan data \textit{testing}; user dapat memilih penggunaan dan jenis setiap atribut pada file yang di-input-kan oleh user.

\begin{figure}[H]
	\centering
	\includegraphics[scale=0.4]{ImplGUI/input}
	\caption[Implementasi Antarmuka Input Set Manager]{Implementasi Antarmuka Input Set Manager}
	\label{fig:Implementasi Antarmuka Input Set Manager}
\end{figure}


\subsubsection{Renew Model Manager}
Gambar~\ref{fig:Implementasi Antarmuka Renew Model Manager} menunjukkan antarmuka untuk melakukan pembaharuan model NBC pada perangkat lunak untuk melakukan testing dan klasifikasi pada modul yang berbasis \textit{non-MapReduce}. Seperti yang sudah dijelaskan di bagian perancangan antarmuka \textit{Renew Model Manager} pada bagian~\ref{subsec:Renew Model Manager}, user bisa melakukan pembaharuan model NBC dari penyimpanan local milik user, atau langsung dari model direktori yang berada di dalam HDFS.

\begin{figure}[H]
	\centering
	\includegraphics[scale=0.45]{ImplGUI/renewmodel}
	\caption[Implementasi Antarmuka Renew Model Manager]{Implementasi Antarmuka Renew Model Manager}
	\label{fig:Implementasi Antarmuka Renew Model Manager}
\end{figure}


\subsubsection{Testing Manager}
Gambar~\ref{fig:Implementasi Antarmuka Testing Manager} menunjukkan antarmuka untuk melakukan testing pada model NBC yang sudah ada di dalam program sebelumnya. Seperti yang sudah dijelaskan di bagian perancangan antarmuka \textit{Testing Manager} pada bagian~\ref{subsec:Testing Manager}, user dapat melakukan testing menggunakan data testing yang berada pada penyimpanan local milik user, atau melalui data testing yang sudah ada pada model direktori dalam HDFS.

\begin{figure}[H]
	\centering
	\includegraphics[scale=0.45]{ImplGUI/testing}
	\caption[Implementasi Antarmuka Testing Manager]{Implementasi Antarmuka Testing Manager}
	\label{fig:Implementasi Antarmuka Testing Manager}
\end{figure}

\subsubsection{Classification Manager}
Gambar~\ref{fig:Implementasi Antarmuka Classification Manager} menunjukkan antarmuka untuk melakukan klasifikasi pada model NBC(\textit{single case prediction}). Seperti yang sudah dijelaskan di bagian perancangan antarmuka \textit{Classification Manager} pada bagian~\ref{subsec:Classification Manager}, user dapat melakukan input manual untuk 1 kasus baru dengan mengisi nilai atribut prediktor yang dibutuhkan dan atribut kelas (\textit{optional}) yang nantinya diumpankan ke dalam model NBC untuk dilakukan klasifikasi.

\begin{figure}[H]
	\centering
	\includegraphics[scale=0.425]{ImplGUI/classification}
	\caption[Implementasi Antarmuka Classification Manager]{Implementasi Antarmuka Classification Manager}
	\label{fig:Implementasi Antarmuka Classification Manager}
\end{figure}

\subsubsection{Error Rate Dashboard}
Gambar~\ref{fig:Implementasi Antarmuka Error Rate Dashboard} menunjukkan antarmuka untuk melakukan monitor terhadap nilai - nilai \textit{error rate} yang muncul setelah melakukan testing terhadap model NBC. Seperti yang sudah dijelaskan di bagian perancangan antarmuka \textit{Error Rate Dashboard} pada bagian~\ref{subsec:Error Rate Dashboard}, user dapat melakukan monitor terhadap matrix \textit{confusion} yang dihasilkan dan juga beberapa nilai - nilai \textit{error rate} yang dihasilkan. Selain itu, untuk mengingatkan user kepada formula - formula yang digunakan untuk menghasilkan nilai \textit{error rate}, pada bagian paling bawah antarmuka juga menampilkan formula - formula yang digunakan dalam menghasilkan tiap nilai \textit{error rate} yang digunakan. 

\begin{figure}[H]
	\centering
	\includegraphics[scale=0.45]{ImplGUI/error_rate}
	\caption[Implementasi Antarmuka Error Rate Dashboard]{Implementasi Antarmuka Error Rate Dashboard}
	\label{fig:Implementasi Antarmuka Error Rate Dashboard}
\end{figure}

\section{Implementasi Package, Kelas, dan Method dengan Java}
\label{sec:impl_code}

Implementasi untuk setiap modul dari kelas-kelas dan method-method hasil dari analisis dan rancangan pada bahasa pemrograman Java yang dikelompokan berdasarkan Package, dicantumkan dan dapat dilihat pada lampiran - lampiran sebagai berikut:
\begin{enumerate}
	\item Modul \textit{Training Naive Bayes M-R Based} berada pada lampiran~\ref{lamp:A}
	\item Modul \textit{Testing Naive Bayes M-R Based} berada pada lampiran~\ref{lamp:B}
	\item Modul Kelola Input dan Klasifikasi berbasis \textit{non-MapReduce} berapa pada lampiran~\ref{lamp:C}
\end{enumerate}

\section{Pengujian Kebenaran}
\label{sec:Pengujian Kebenaran}
Bagian ini menunjukkan pengujian kebenaran terhadap perangkat lunak yang dibangun dengan membandingkan perhitungan manual yang dilakukan pada bagian~\ref{subsec:Perhitungan Manual Dengan Data Studi Kasus} dengan perhitungan dengan menggunakan program pada bagian~\ref{subsec:Perhitungan Menggunakan Perangkat Lunak yang Dibangun Dengan Data Studi Kasus} dengan data yang sama. Data yang digunakan pada kedua perhitungan tersebut merupakan data yang sama. Data studi kasus diperoleh dengan cara mengambil dari situs \url{http://cse-wiki.unl.edu/wiki/images/e/e0/Golf_dataset.png} yang diambil pada pada 23 April 20117, dan telah dilakukan modifikasi sedikit terhadap data agar menjadi lebih simpel.

\subsection{Perhitungan Manual Dengan Data Studi Kasus}
\label{subsec:Perhitungan Manual Dengan Data Studi Kasus}
Pada bagian ini dilakukan perhitungan manual dengan menggunakan data studi kasus yang menunjukkan seseorang bermain tenis atau tidak berdasarkan dari data kelembaban dari cuaca dan pemandangan yang terjadi.

\subsubsection{Pembuatan Model NBC}
%\subsubsection{Tahapan Membuat Model Klasifikasi Dari Algoritma Naive Bayes}		
Misal kita memiliki dataset yang menunjukan seseorang bermain tenis atau tidak berdasarkan dari data kelembaban dan pemandangan yang terjadi seperti pada Tabel~\ref{tab:dataset} berikut: 
		
		\begin{table}[H]
		\label{tab:dataset}
		\centering
		\caption{Contoh Dataset (atribut kelas = \textbf{Play})}
		\begin{tabular}{ | c | c | c | }
		\hline
		%\toprule
		 Humidity & Outlook & \textbf{Play}\\ \hline \hline
		%\midrule
		60 & Rainy & No\\ \hline
		%\midrule
		78 & Rainy & No\\ \hline
		%\midrule
		80 & Sunny & Yes\\ \hline
		%\midrule
		75 & Sunny & No\\ \hline
		%\midrule
		85 & Sunny & Yes \\ \hline
		%\bottomrule
		\end{tabular}
		\end{table}
		
		
		Langkah pertama yang perlu dilakukan untuk membuat algoritma naive bayes classifier adalah membuat table frekuensi untuk setiap atribut prediktor terhadap atribut kelas yang bertipe Diskrit. Pada contoh tabel diatas, diasumsikan bahwa atribut Humidity dan Outlook merupakan atribut prediktor, lalu untuk atribut Play merupakan atribut kelas.
		
		\begin{table}[ht]
			\centering
			\caption{Table frekuensi atribut Outlook}
			\begin{tabular}{ | c | c | c | c | }
			\hline
			 & \multicolumn{2}{c}{\textbf{Play}} & \\ 
			%\hline
			 & Yes & No & \textit{sum} \\
			\hline
			Sunny & 2 & 1 & \textbf{3}\\
			\hline
			Rainy & 0 & 2 & \textbf{2} \\
			\hline
			\textit{sum} & \textbf{2} & \textbf{3} & \\
			\hline
			\end{tabular}
		\end{table}
		
		Pada table frekuensi untuk atribut Outlook, dapat dilihat bahwa $P(X=Rainy|C=Yes) = 0$. Naive bayes classifier tidak dapat mengatasi frekuensi yang nilainya 0. Karena dapat menyebabkan seluruh perhitungan menjadi 0(karena berapapun bilangannya, jika dikalikan dengan 0 selalu menghasilkan nilai 0), sehingga menjadi tidak relevan. Berdasarkan pada~\ref{subsec:Zero-Frequency Problem}, perubahan nilai atribut dapat dilihat pada table 3. \\ 

		\begin{table}[ht]
			\centering
			\caption{Table frekuensi atribut Outlook}
			\begin{tabular}{|c|c|c|c|}
			\hline
			 & \multicolumn{2}{c}{\textbf{Play}} & \\
			 & Yes & No & \textit{sum} \\ 
			\hline
			Sunny & 3 & 2 & \textbf{5}\\
			\hline
			Rainy & 1 & 3 & \textbf{4} \\
			\hline
			\textit{sum} & \textbf{4} & \textbf{5} & \\
			\hline
			\end{tabular}
		\end{table}
		
		
		Langkah kedua adalah membuat table kemungkinan dari table frekuensi yang telah dibuat :		
		\begin{table}[ht]
			\centering
			\caption{Table kemungkinan atribut Outlook}
			\begin{tabular}{|c|c|c|c|}
			\toprule
			 & \multicolumn{2}{c}{\textbf{Play}} & \\
			 & Yes & No & \textit{sum} \\ 
			\midrule
			Sunny & 3/4 & 2/5 & \textbf{5/9}\\
			\midrule
			Rainy & 1/4 & 3/5 & \textbf{4/9} \\
			\midrule
			\textit{sum} & \textbf{4/9} & \textbf{5/9} & \\
			\bottomrule
			\end{tabular}
		\end{table}
		
		Karena atribut Humidity bertipe numerik, atribut tersebut perlu diubah ke dalam kategori mereka masing - masing agar perhitungan dalam pembuatan model dapat tepat. Konversi atribut yang bertipe numerik bisa menggunakan distribusi variabel numerik untuk dapat menebak frekuensi-nya dengan mengasumsikan distribusi normal untuk variabel numerik. Rumus yang digunakan adalah :\\
		
		\textbf{Mencari mean (rata - rata)} 
		\begin{equation}
			\mu = \dfrac{1}{n} \mathlarger{\mathlarger{\sum}}_{i=1}^{n}x_i
		\end{equation}
		
		\textbf{Mencari Standard Deviation}
		\begin{equation}
			\sigma = \mathlarger{[ \dfrac{1}{n - 1} * \mathlarger{\mathlarger{\sum}}_{i=1}^{n} (x_i - \mu)^2]}^{0.5}
		\end{equation}
		
		\textbf{Normal Distribution}
		\begin{equation}
			f(x) = \dfrac{1}{\sqrt{2\pi\sigma}}e^{-\dfrac{(x-\mu)^2}{2\sigma^2}}
		\end{equation}
		
		Berikut merupakan table rata - rata dan standar deviasi dari atribut Humidity yang bertipe numerik : 
		
		\begin{table}[h]
		\centering
		\caption{Table rata - rata dan standar deviasi atribut Humidity}
		\begin{tabular}{|c|c|c|c|c|}
		\toprule
		\multirow{2}{*}{Play Golf ?} & Yes & 80 & 85 & \\
		 & No & 60 & 75 & 78 \\
		\bottomrule
		\end{tabular}
		\end{table}
		
		\begin{table}[ht]
		\centering
		\caption{Table Distribusi}
		\begin{tabular}{|c|c|c|}
		\toprule
		 & Mean & StDev \\
		Yes & 82.5 & 3.5 \\
		No & 71 & 9.6 \\
		\bottomrule
		\end{tabular}
		\end{table}
		
		Dari table distribusi tersebut didapatkan formula untuk menghitung klasifikasi untuk atribut Humidity adalah:
		
		\begin{equation}
			%f(x|play=yes) = \dfrac{1}{\sqrt{2\pi(3.5)}}e^{-\dfrac{(x-82.5)^2}{2(3.5)^2}} 
			f(x|play=yes) = \dfrac{1}{\sqrt{2\pi(3.5)}}e^{-\dfrac{(x-82.5)^2}{2(3.5)^2}}
		\end{equation}
		\begin{equation}
			f(x|play=no) = \dfrac{1}{\sqrt{2\pi(9.6)}}e^{-\dfrac{(x-71)^2}{2(9.6)^2}} 
		\end{equation}
		
		Setelah semua model dari naive bayes classifier telah jadi, maka klasifikasi sudah dapat dilakukan dengan model diatas.

\subsubsection{Klasifikasi Menggunakan Model NBC}
%\subsection{Contoh perhitungan klasifikasi algoritma naive bayes}		
		Dimisalkan kita memiliki 2 buah dataset yang diuji menggunakan model klasifikasi yang telah dibangun sebelumnya, seperti berikut :
		
\begin{enumerate}
	\item $X = {Humidity = 50, Outlook = Sunny}$
	\item $Y = {Humidity = 90, Outlook = Sunny}$
\end{enumerate}
		
Untuk dataset X dan Y, dicari peluang kelas yang paling tinggi. \\
		
$( C_{MAP} = \underset{c \in C}{ argmax } P(c|d) = \underset{c \in C}{ argmax } \dfrac{P(d|c) P(c)}{P(d)} = \underset{c \in C}{ argmax } P(d|c) P(c) )$
		
\paragraph{Untuk dataset X dengan $P=Yes$:}
	Menghitung peluang untuk atribut $Outlook=Sunny$ dengan $P=Yes$
	\begin{equation}
			P(Outlook=Sunny|Yes) = 3/4 
			= 0.75
	\end{equation}
	
	Menghitung peluang untuk atribut $Humidity=50$ dengan $P=Yes$
		\begin{equation}
			P(Humidity=50|Yes) 
			= \dfrac{1}{\sqrt{2\pi(3.5)}}e^{-\dfrac{(\textbf{50}-82.5)^2}{2(3.5)^2}}
			= 4.031*10^{-20}
		\end{equation}
	
	\paragraph{Untuk dataset X dengan $P=No$:}
		Menghitung peluang untuk atribut $Outlook=Sunny$ dengan $P=No$
		\begin{equation}
			P(Outlook=Sunny|No) = 2/5 
			= 0.4
		\end{equation}
		Menghitung peluang untuk atribut $Humidity=50$ dengan $P=No$
		\begin{equation}
			P(Humidity=50|No) 
			= \dfrac{1}{\sqrt{2\pi(9.6)}}e^{-\dfrac{(\textbf{50}-71)^2}{2(9.6)^2}}
			= 0.011
		\end{equation}
	\paragraph{Kesimpulan untuk $dataset$ $X$}
	Dari perhitungan di atas, didapat bahwa : \\ \\
	Untuk kelas $Play=Yes$ \\
	$P(Play=Yes|X) \\
	= P(Outlook=Sunny|Play=Yes)*P(Humidity=50|Play=Yes)*P(Yes) \\
	= 0.75 * 4.031*10^{-20} * 4/9 \\
	= 1.343*10^{-20}$ \\ \\
	Untuk kelas $Play=No$ \\
	$P(Play=No|X) \\
	= P(Outlook=Sunny|Play=No)*P(Humidity=50|Play=No)*P(No) \\
	= 0.4 * 0.011 * 5/9 \\
	= 0.002$ \\ \\
	Setelah itu, lakukan normalisasi terhadap nilai - nilai berikut: \\
	$P(Play=Yes|X) = 1.343*10^{-20} / (1.343*10^{-20} + 0.002) \\
	= 6.715*10^{-18}$ \\
	$P(Play=No|X) = 0.002 / ((1.343*10^{-20}) + 0.002) \\
	= 1 (100\%)$ \\
	
	Karena, $P(Play=Yes|X) = 6.715*10^{-18} < P(Play=No|X) = 1 (100\%) $, maka hasil klasifikasi untuk \textit{dataset X} ialah kelas $Play=No$.
	
	\paragraph{Untuk dataset Y dengan $P=Yes$}
	Menghitung peluang untuk atribut $Outlook=Sunny$ dengan $P=Yes$
	\begin{equation}
			P(Outlook=Sunny|Yes) = 3/4 
			= 0.75
	\end{equation}
	
	Menghitung peluang untuk atribut $Humidity=90$ dengan $P=Yes$
		\begin{equation}
			P(Humidity=90|Yes) 
			= \dfrac{1}{\sqrt{2\pi(3.5)}}e^{-\dfrac{(\textbf{90}-82.5)^2}{2(3.5)^2}}
			= 0.021
		\end{equation}
	
	\paragraph{Untuk dataset Y dengan $P=No$:}
	Menghitung peluang untuk atribut $Outlook=Sunny$ dengan $P=No$
	\begin{equation}
				P(Outlook=Sunny|No)
				= 2/5 
				= 0.4
		\end{equation}
	
	Menghitung peluang untuk atribut $Humidity=90$ dengan $P=No$
		\begin{equation}
			P(Humidity=90|No)
			= \dfrac{1}{\sqrt{2\pi(9.6)}}e^{-\dfrac{(\textbf{90}-71)^2}{2(9.6)^2}} 
			= 0.018
		\end{equation}
		
	\paragraph{Kesimpulan untuk $dataset$ $Y$}
	Dari perhitungan di atas, didapat bahwa : \\ \\
	Untuk kelas $Play=Yes$ \\
	$P(Play=Yes|Y) \\
	= P(Outlook=Sunny|Play=Yes)*P(Humidity=90|Play=Yes)*P(Yes) \\
	= 0.75 * 0.021 * 4/9 \\
	= 0.007$ \\ \\
	Untuk kelas $Play=No$ \\
	$P(Play=No|Y) \\
	= P(Outlook=Sunny|Play=No)*P(Humidity=90|Play=No)*P(No) \\
	= 0.4 * 0.018 * 5/9 \\
	= 0.004$ \\ \\
	Setelah itu, lakukan normalisasi terhadap nilai - nilai berikut: \\
	$P(Play=Yes|Y) = 0.007 / (0.007+0.004) \\
	= 0.64 (64\%)$ \\
	$P(Play=No|Y) = 0.004 / (0.007+0.004) \\
	= 0.46 (46\%)$ \\
	
	Karena, $P(Play=Yes|Y) = 0.64 (64\%) > P(Play=No|Y) = 0.46 (46\%)$, maka hasil klasifikasi untuk \textit{dataset Y} ialah kelas $Play=Yes$.
	

\subsection{Perhitungan Menggunakan Perangkat Lunak yang Dibangun Dengan Data Studi Kasus}
\label{subsec:Perhitungan Menggunakan Perangkat Lunak yang Dibangun Dengan Data Studi Kasus}

Pada bagian ini dilakukan perhitungan manual dengan menggunakan data studi kasus yang menunjukkan seseorang bermain tenis atau tidak berdasarkan dari data kelembaban dari cuaca dan pemandangan yang terjadi.

\subsubsection{Pembuatan model NBC}
\label{subsubsec:Pembuatan model naive bayes classifier}

Untuk dataset seperti berikut:
\begin{table}[H]
\label{tab:dataset-pl}
\centering
\caption{Contoh Dataset (atribut kelas = \textbf{Play})}
\begin{tabular}{ | c | c | c | }
\hline
%\toprule
	Humidity & Outlook & \textbf{Play}\\ \hline \hline
%\midrule
60 & Rainy & No\\ \hline
%\midrule
78 & Rainy & No\\ \hline
%\midrule
80 & Sunny & Yes\\ \hline
%\midrule
75 & Sunny & No\\ \hline
%\midrule
85 & Sunny & Yes \\ \hline
%\bottomrule
\end{tabular}
\end{table}

Akan dilakukan eksekusi program bebasis \textit{MapReduce} untuk melakukan training pembuatan model NBC dengan melakukan perintah (\textit{*diasumsikan data input telah berada pada direktori \texttt{/bayes/weather/input} di dalam HDFS}):
\begin{lstlisting}
:~$ hadoop jar mapreduce-train.jar /bayes/weather
\end{lstlisting}

Keterangan hasil output dari \textit{log} yang diterima lewat terminal setelah melakukan eksekusi program dapat dilihat pada lampiran~\ref{lamp:D}

Hasil pembuatan model klasifikasi NBC yang dilakukan oleh program dapat dilihat pada direktori \texttt{/bayes/weather/output/part-r-00000}. Hasil model tersebut adalah sebagai berikut:
\begin{lstlisting}
Play,No,3.0|CLASS	
Play,Yes,2.0|CLASS	
Humidity,Play,No	;71.0|9.643650760992955|NUMERIC
Humidity,Play,Yes	;82.5|3.5355339059327378|NUMERIC
Outlook,Rainy,Play,No,2.0|DISCRETE	
Outlook,Sunny,Play,No,1.0|DISCRETE	
Outlook,Sunny,Play,Yes,2.0|DISCRETE
\end{lstlisting}

Untuk model NBC pada atribut diskrit dan kelas, model yang dihasilkan memiliki nilai yang berbeda dengan model yang dihasilkan pada perhitungan manual. Hal ini disebabkan oleh karena penanganan \textit{zero-frequency problem} pada perangkat lunak tidak ditangani pada saat pembuatan model, melainkan pada saat melakukan testing dan klasifikasi.

\subsubsection{Klasifikasi Menggunakan Model NBC}
Setelah model NBC diperoleh dari hasil eksekusi program \textit{train naive bayes} berbasis \textit{MapReduce} pada bagian~\ref{subsubsec:Pembuatan model naive bayes classifier}, dilakukan klasifikasi dan testing pada model NBC tersebut.

Pada bagian ini, dilakukan 2 kali pengujian klasifikasi, yaitu (1) dengan menggunakan perangkat lunak berbasis \textit{MapReduce} pada modul \textit{Testing Naive Bayes} dan (2) dengan menggunakan perangkat lunak berbasis \textit{non-MapReduce} pada modul Klasifikasi.
\begin{enumerate}
	\item{Klasifikasi dengan perangkat lunak berbasis \textit{MapReduce}}\\
	Diasumsikan data \textit{testing} untuk pengujian model NBC yang sama dengan pada perhitungan manual sudah berada pada direktori \texttt{/bayes/weather/testing/input} dalam HDFS. Program testing dieksekusi dengan menjalankan perintah:
	\begin{lstlisting}
	:~$ hadoop jar mapreduce-train.jar /bayes/weather
	\end{lstlisting}
	
	Keterangan hasil output dari \textit{log} yang diterima lewat terminal setelah melakukan eksekusi program dapat dilihat pada lampiran~\ref{lamp:D}.
	
	Setelah berhasil, dapat dilihat hasil dari output program pada direktori\\ \texttt{/bayes/weather/testing/output/part-r-00000}:
	\begin{lstlisting}
	@play
	####
	-------------------
	|     | no  | yes |
	-------------------
	| no  | 0   | 0   |
	| yes | 1   | 1   |
	-------------------
	####
	\end{lstlisting}
	Hasil pada program berbasis \textit{MapReduce} dapat dilihat pada hasil matrix \textit{confusion} diatas. Pada matrix \textit{confusion} 2 dimensi berikut, dimensi pertama (mendatar) menyatakan nilai sebenarnya(\textit{actual}) dari data asli-nya (di-input oleh user), dan dimensi kedua (menurun) menyatakan nilai hasil dari prediksi menggunakan model NBC yang sudah dibuat.
	
	Pada matrix tersebut terlihat bahwa terdapat 1 record yang menghasilkan kelas \texttt{Play=Yes} dan 1 record yang menghasilkan kelas \texttt{Play=No}.
	
	\item{Klasifikasi dengan perangkat lunak berbasis \textit{non-MapReduce}}\\
	Model pada NBC di-\textit{import} ke dalam program yang dibuat dengan melakukan pembaharuan model pada antarmuka \textit{Renew Model Manager}. Program melakukan pembacaan terhadap model NBC dan mendeteksi dari model NBC yang diperoleh sebelumnya terdapat \textit{zero-frequency problem} dan mengatasinya dengan menggunakan metode \textit{laplacian correction}, dan hasilnya adalah seperti berikut:
	\begin{figure}[H]
	\centering
	\includegraphics[scale=0.5]{Pengujian/pengujian-nonmr}
	\caption[Import Model NBC - Pengujian]{Import Model NBC - Pengujian}
	\label{fig:Import Model NBC - Pengujian}
	\end{figure}
Dapat dilihat pada Gambar~\ref{fig:Import Model NBC - Pengujian} bahwa pada model yang dihasilkan sebelumnya kita tidak memiliki frekuensi kemunculan untuk prediktor \texttt{outlook=rainy} dengan kelas \texttt{play=yes}, tetapi setelah dilakukan import model ke dalam program, program telah menangani \textit{zero-frequency problem} dengan baik, dengan menambahkan frekuensi sebanyak 1 untuk tiap model.\\

Untuk data input klasifikasi \texttt{Humidity=50} dan \texttt{Outlook=Sunny}, diperoleh hasil klasifikasi yang sama dengan perhitungan manual, yaitu kelas \texttt{Play=No}, dengan presentase dominasi yang juga sama, yaitu sebesar \texttt{100\%}.
\begin{figure}[H]
	\centering
	\includegraphics[scale=0.5]{Pengujian/result_classify_1}
	\caption[Hasil Klasifikasi Dataset-1]{Hasil Klasifikasi Dataset-1}
	\label{fig:Hasil Klasifikasi Dataset-1}
\end{figure}

Untuk data input klasifikasi \texttt{Humidity=90} dan \texttt{Outlook=Sunny}, diperoleh hasil klasifikasi yang sama dengan perhitungan manual, yaitu kelas \texttt{Play=Yes}, dengan presentase dominasi yang berbeda sebesar \texttt{64.67\%}.
\begin{figure}[H]
	\centering
	\includegraphics[scale=0.5]{Pengujian/result_classify_2}
	\caption[Hasil Klasifikasi Dataset-1]{Hasil Klasifikasi Dataset-2}
	\label{fig:Hasil Klasifikasi Dataset-2}
\end{figure}
	
\end{enumerate}

\subsection{Perbandingan dan Kesimpulan}
Pada Subbab ini, dijelaskan perbandingan dan kesimpulan dari hasil pembuatan model dan klasifikasi pada pengujian perhitungan manual dan pengujian program.

\subsubsection{Model}

% Model
\begin{figure}[H]
	\centering
	\includegraphics[scale=0.65]{GambarIO/perbandingan_model}
	\caption[Perbandingan Hasil Model Manual dan Program]{Perbandingan Hasil Model Manual dan Program}
	\label{fig:Perbandingan Hasil Model Manual dan Program}
\end{figure}

Pada hasil pengujian pembuatan model NBC yang dilakukan oleh perhitungan manual dan perangkat lunak, dapat dilihat bahwa model yang dihasilkan oleh perangkat lunak berbeda dengan model yang dihasilkan pada perhitungan manual. Untuk atribut diskrit, perbedaan jumlah frekuensi disebabkan oleh karena penanganan \textit{zero-frequency problem} pada perangkat lunak ditangani pada saat melakukan \textit{testing} program atau pada saat melakukan klasifikasi. Selain itu, perbedaan pada atribut numerik yang bernilai sangat kecil, disebabkan oleh perbedaan angka pembulatan pada perhitungan manual dan perhitungan menggunakan perangkat lunak.

\subsubsection{Klasifikasi}

\begin{figure}[H]
	\centering
	\includegraphics[scale=0.65]{GambarIO/perbandingan_klasifikasi}
	\caption[Perbandingan Hasil Klasifikasi Manual dan Program]{Perbandingan Hasil Klasifikasi Manual dan Program}
	\label{fig:Perbandingan Hasil Klasifikasi Manual dan Program}
\end{figure}

% Klasifikasi
Pada hasil pengujian klasifikasi yang telah dilakukan, dapat dibuktikan bahwa hasil perhitungan dengan menggunakan program dan manual memiliki hasil yang sama. Hanya saja angka presentase dominasi hasil yang dimiliki oleh program sedikit berbeda dengan angka presentase dominasi pada perhitungan manual. Untuk dataset pertama, pada perhitungan manual dan menggunakan program diperoleh hasil kelas yang sama yaitu \texttt{Play=Yes} dengan tingkat presentase dominasi juga sama sebesar \texttt{100\%}. Untuk dataset pertama, pada perhitungan manual diperoleh hasil kelas \texttt{Play=Yes} dengan tingkat presentase dominasi sebesar \texttt{64\%}, sedangkan jika menggunakan program tingkat presentase dominasi-nya sebesar \texttt{64.67\%}. Perbedaan yang sangat kecil, yang berkisar diantara \texttt{0.6\%} atau \texttt{0.006} diperkirakan timbul karena pembulatan nilai desimal yang berbeda.

\section{Eksperimen}
Bagian ini memaparkan mengenai eksperimen - eksperimen yang dilakukan dalam membuat model NBC dan melakukan klasifikasi.

\subsection{Pembuatan model}
Pada eksperimen pembuatan model, digunakan 3 dataset yang berbeda.

\subsubsection{Dataset-1}
Dataset-1: \textit{mushroom classification}, berisi mengenai data tentang tumbuhan jamur yang memiliki racun(tidak dapat dimakan) dan yang tidak memiliki racun(bisa dimakan) berdasarkan beberapa atribut yang dimiliki oleh jamur tersebut. Data ini diperoleh dari \url{https://www.kaggle.com/uciml/mushroom-classification} pada tanggal 26 April 2017(Contoh dataset telah dilampirkan pada lampiran~\ref{lamp:E-Contoh Datase1-1}). Data ini memiliki 23 atribut, dimana 1 merupakan atribut kelas yang menentukan jamur tersebut beracun atau tidak dan sisanya merupakan atribut dari ciri - cirijamur itu sendiri. Berikut merupakan penjelasan tiap atribut milik jamur yang disingkat menjadi 1 huruf:
\begin{enumerate}
	\item \texttt{classes : edible=e, poisonous=p} (jumlah nilai unik = 2) \textcolor{red}{*}\texttt{atribut kelas}
	\item \texttt{cap-shape: bell=b, conical=c, convex=x, flat=f, knobbed=k, sunken=s} (jumlah nilai unik = 6) \textcolor{red}{*}\texttt{atribut prediktor}
	\item \texttt{cap-surface: fibrous=f, grooves=g, scaly=y, smooth=s} (jumlah nilai unik = 4) \textcolor{red}{*}\texttt{atribut prediktor}
	\item \texttt{cap-color: brown=n, buff=b, cinnamon=c, gray=g, green=r, pink=p, purple=u, red=e, white=w, yellow=y} (jumlah nilai unik = 10) \textcolor{red}{*}\texttt{atribut prediktor}
	\item \texttt{bruises: bruises=t, no=f} (jumlah nilai unik = 2) \textcolor{red}{*}\texttt{atribut prediktor}
	\item \texttt{odor: almond=a, anise=l, creosote=c, fishy=y, foul=f, musty=m, none=n, pungent=p, spicy=s} (jumlah nilai unik = 9) \textcolor{red}{*}\texttt{atribut prediktor}
	\item \texttt{gill-attachment: attached=a, descending=d, free=f, notched=n} (jumlah nilai unik = 4) \textcolor{red}{*}\texttt{atribut prediktor}
	\item \texttt{gill-spacing: close=c, crowded=w, distant=d} (jumlah nilai unik = 3) \textcolor{red}{*}\texttt{atribut prediktor}
	\item \texttt{gill-size: broad=b, narrow=n} (jumlah nilai unik = 2) \textcolor{red}{*}\texttt{atribut prediktor}
	\item \texttt{gill-color: black=k, brown=n, buff=b, chocolate=h, gray=g ,green=r, orange=o, pink=p, purple=u, red=e, white=w, yellow=y} (jumlah nilai unik = 12) \textcolor{red}{*}\texttt{atribut prediktor}
	\item \texttt{stalk-shape: enlarging=e, tapering=t} (jumlah nilai unik = 3) \textcolor{red}{*}\texttt{atribut prediktor}
	\item \texttt{stalk-root: bulbous=b, club=c, cup=u, equal=e, rhizomorphs=z, rooted=r, missing=?} (jumlah nilai unik = 7) \textcolor{red}{*}\texttt{atribut prediktor}
	\item \texttt{stalk-surface-above-ring: fibrous=f, scaly=y, silky=k, smooth=s} (jumlah nilai unik = 4) \textcolor{red}{*}\texttt{atribut prediktor}
	\item \texttt{stalk-surface-below-ring: fibrous=f, scaly=y, silky=k, smooth=s} (jumlah nilai unik = 4) \textcolor{red}{*}\texttt{atribut prediktor}
	\item \texttt{stalk-color-above-ring: brown=n, buff=b, cinnamon=c, gray=g, orange=o, pink=p, red=e, white=w, yellow=y} (jumlah nilai unik = 9) \textcolor{red}{*}\texttt{atribut prediktor}
	\item \texttt{stalk-color-below-ring: brown=n, buff=b, cinnamon=c, gray=g, orange=o, pink=p, red=e, white=w, yellow=y} (jumlah nilai unik = 9) \textcolor{red}{*}\texttt{atribut prediktor}
	\item \texttt{veil-type: partial=p, universal=u} (jumlah nilai unik = 2) \textcolor{red}{*}\texttt{atribut prediktor}
	\item \texttt{veil-color: brown=n, orange=o, white=w, yellow=y} (jumlah nilai unik = 4) \textcolor{red}{*}\texttt{atribut prediktor}
	\item \texttt{ring-number: none=n, one=o, two=t} (jumlah nilai unik = 2) \textcolor{red}{*}\texttt{atribut prediktor}
	\item \texttt{ring-type: cobwebby=c, evanescent=e, flaring=f, large=l, none=n, pendant=p, sheathing=s, zone=z} (jumlah nilai unik = 8) \textcolor{red}{*}\texttt{atribut prediktor}
	\item \texttt{spore-print-color: black=k, brown=n, buff=b, chocolate=h, green=r, orange=o, purple=u, white=w, yellow=y} (jumlah nilai unik = 9) \textcolor{red}{*}\texttt{atribut prediktor}
	\item \texttt{population: abundant=a ,clustered=c, numerous=n, scattered=s, several=v, solitary=y} (jumlah nilai unik = 6) \textcolor{red}{*}\texttt{atribut prediktor}
	\item \texttt{habitat: grasses=g, leaves=l, meadows=m, paths=p, urban=u, waste=w, woods=d} (jumlah nilai unik = 7) \textcolor{red}{*}\texttt{atribut prediktor}
\end{enumerate}

Dataset-1 asli ini memiliki ukuran sebesar 366KB. Pada eksperimen kali ini dilakukan manipulasi terhadap dataset-1 agar menjadi lebih besar. Manipulasi data dilakukan dengan cara melakukan pengulangan terhadap data yang sama hingga mencapai ukuran 174,9MB. Hal ini dilakukan agar data yang diumpankan kepada program mapreduce tidak terlalu kecil. Pada dataset-1 yang berukuran 174,9MB, eksperimen dilakukan dengan presentase data training dan data testing sebesar 70 : 30. Selain itu, eksperimen juga mengikutsertakan seluruh atribut prediktor pada dataset ini dan 1 atribut kelas.

Data yang sudah diperbanyak tersebut dimasukkan ke dalam HDFS melalui perangkat lunak pada modul input seperti pada gambar~\ref{fig:Memasukkan dataset-1 ke dalam HDFS}

\begin{figure}[H]
	\centering
	\includegraphics[scale=0.65]{Eksperimen_Model/dataset1_0}
	\caption[Memasukkan dataset-1 ke dalam HDFS]{Memasukkan dataset-1 ke dalam HDFS}
	\label{fig:Memasukkan dataset-1 ke dalam HDFS}
\end{figure}

Setelah melakukan eksekusi program untuk melakukan pembuatan model dengan menggunakan 70\% dari dataset-1, diperoleh hasil model NBC yang tertera pada lampiran~\ref{lamp:E-Hasil Pembuatan Model Dataset-1}.

Setelah model NBC dihasilkan, dilakukan juga eksperimen testing untuk mengetahui kualitas dari NBC yang dihasilkan dengan menggunakan 30\% dari dataset-1, diperoleh hasil testing yang tertera pada lampiran~\ref{lamp:E-Hasil Testing Dataset-1}. Pada lampiran tersebut, dapat dilihat bahwa model NBC yang dihasilkan memiliki tingkat akurasi yang cukup tinggi. Hal tersebut dapat terlihat dari hasil \textit{confusion matrix} dan nilai perhitungan evaluasi model NBC yang tertera pada Tabel~\ref{tab:confusion matrix - mushroom} dan Tabel~\ref{tab:model evaluation - mushroom}. Pada nilai \texttt{accuracy} yang mencapai nilai \texttt{99\%} dan nilai \textit{recall}, \textit{precision}, dan \textit{f-measure} untuk setiap nilai kelas yang juga memiliki nilai sebesar \texttt{99\%}.

\begin{table}[H]
\label{tab:confusion matrix - mushroom}
\centering
\caption{\textit{Confusion Matrix - mushroom classification}}
\begin{tabular}{ | c | c | c | }
\hline
& \textbf{e} & \textbf{p}\\ \hline \hline
\textbf{e} & 567548 & 2720 \\ \hline
\textbf{p} & 5576 & 526746 \\ \hline
\end{tabular}
\end{table}

\begin{table}[H]
\label{tab:model evaluation - mushroom}
\centering
\caption{Hasil Evaluasi Model NBC - \textit{mushroom classification}}
\begin{tabular}{ | c | c | c | c | c | }
\hline
\textit{\textbf{Class}} & \textit{\textbf{Accuracy}} & \textit{\textbf{Precision}} & \textit{\textbf{Recall}} & \textit{\textbf{F-Measure}}\\ \hline \hline
\textbf{p} & 0.992 & 0.994 & 0.989 & 0.994 \\ \hline
\textbf{e} & 0.992 & 0.990 & 0.995 & 0.990 \\ \hline
\end{tabular}
\end{table}

\subsubsection{Dataset-2}

Dataset-2: \textit{Car Evaluation Data Set}, berisi mengenai data dari evaluasi mobil yang akan menentukkan apakah sebuah mobil dapat diterima atau tidak. Data ini diperoleh dari \url{http://mlr.cs.umass.edu/ml/datasets/Car+Evaluation} pada tanggal 29 April 2017(Contoh dataset telah dilampirkan pada lampiran~\ref{lamp:E-Contoh Datase1-2}). Data ini memiliki 7 atribut, dimana 1 atribut merupakan atribut kelas dan sisanya adalah atribut prediktor. Berikut merupakan penjelasan tiap atribut yang diperoleh dari \url{http://mlr.cs.umass.edu/ml/datasets/Car+Evaluation}:
\begin{enumerate}
	\item \texttt{class: unacc, acc, good, vgood } (jumlah nilai unik = 4) \textcolor{red}{*}\texttt{atribut kelas}
	\item \texttt{buying: vhigh, high, med, low} (jumlah nilai unik = 4) \textcolor{red}{*}\texttt{atribut prediktor}
	\item \texttt{maint: vhigh, high, med, low} (jumlah nilai unik = 4) \textcolor{red}{*}\texttt{atribut prediktor}
	\item \texttt{doors: 2, 3, 4, 5more} (jumlah nilai unik = 4) \textcolor{red}{*}\texttt{atribut prediktor}
	\item \texttt{persons: 2, 4, more} (jumlah nilai unik = 3) \textcolor{red}{*}\texttt{atribut prediktor}
	\item \texttt{lug boot: small, med, big} (jumlah nilai unik = 3) \textcolor{red}{*}\texttt{atribut prediktor}
	\item \texttt{safety: low, med, high} (jumlah nilai unik = 3) \textcolor{red}{*}\texttt{atribut prediktor}
\end{enumerate}

Dataset-2 asli ini memiliki ukuran sebesar 53KB. Pada eksperimen kali ini dilakukan manipulasi terhadap dataset-2 agar menjadi lebih besar. Manipulasi data dilakukan dengan cara melakukan pengulangan terhadap data yang sama hingga mencapai ukuran 64.3MB. Hal ini dilakukan agar data yang diumpankan kepada program mapreduce tidak terlalu kecil. Pada dataset-2 yang berukuran 74MB, eksperimen dilakukan dengan presentase data training dan data testing sebesar 70 : 30. Selain itu, eksperimen juga mengikutsertakan seluruh atribut prediktor dan 1 atribut kelas bernama \texttt{class}.

Data yang sudah diperbanyak tersebut dimasukkan ke dalam HDFS melalui perangkat lunak pada modul input seperti pada gambar~\ref{fig:Memasukkan dataset-2 ke dalam HDFS}

\begin{figure}[H]
	\centering
	\includegraphics[scale=0.65]{Eksperimen_Model/dataset2_0}
	\caption[Memasukkan dataset-2 ke dalam HDFS]{Memasukkan dataset-2 ke dalam HDFS}
	\label{fig:Memasukkan dataset-2 ke dalam HDFS}
\end{figure}

Setelah melakukan eksekusi program untuk melakukan pembuatan model dengan menggunakan 70\% dari dataset-2, diperoleh hasil model NBC yang tertera pada lampiran~\ref{lamp:E-Hasil Pembuatan Model Dataset-2}.

Setelah model NBC dihasilkan, dilakukan juga eksperimen testing untuk mengetahui kualitas dari NBC yang dihasilkan dengan menggunakan 30\% dari dataset-2, diperoleh hasil testing yang tertera pada lampiran~\ref{lamp:E-Hasil Testing Dataset-2}. Pada lampiran tersebut, dapat dilihat bahwa model NBC yang dihasilkan memiliki tingkat akurasi yang cukup tinggi. Hal tersebut dapat terlihat dari hasil \textit{confusion matrix} dan nilai perhitungan evaluasi model NBC yang tertera pada Tabel~\ref{tab:confusion matrix - car} dan Tabel~\ref{tab:model evaluation - car}. Pada hasil evaluasi tersebut, dapat dilihat bahwa nilai \texttt{accuracy} mencapai nilai \texttt{87\%} dan nilai \textit{recall}, \textit{precision}, dan \textit{f-measure} untuk setiap nilai kelas yang juga memiliki nilai rata - rata sebesar \texttt{70\%}.

\begin{table}[H]
\label{tab:confusion matrix - car}
\centering
\caption{\textit{Confusion Matrix - car evaluation}}
\begin{tabular}{ | c | c | c | c | c | }
\hline
& \textbf{acc} & \textbf{good} & \textbf{unacc} & \textbf{vgood} \\ \hline \hline
\textbf{acc} & 100572 & 3480 & 29580 & 0 \\ \hline
\textbf{good} & 15660 & 7656 & 0 & 696 \\ \hline
\textbf{unacc} & 16008 & 696 & 404028 & 0 \\ \hline
\textbf{vgood} & 9744 & 0 & 0 & 12529 \\ \hline
\end{tabular}
\end{table}

\begin{table}[H]
\label{tab:model evaluation - car}
\centering
\caption{Hasil Evaluasi Model NBC - \textit{car evaluation}}
\begin{tabular}{ | c | c | c | c | c | }
\hline
\textit{\textbf{Class}} & \textit{\textbf{Accuracy}} & \textit{\textbf{Precision}} & \textit{\textbf{Recall}} & \textit{\textbf{F-Measure}}\\ \hline \hline
\textbf{acc} & 0.873 & 0.708 & 0.752 & 0.719 \\ \hline
\textbf{good} & 0.873 & 0.647 & 0.318 & 0.407 \\ \hline
\textbf{unacc} & 0.873 & 0.931 & 0.960 & 0.933 \\ \hline
\textbf{vgood} & 0.873 & 0.947 & 0.562 & 0.788 \\ \hline
\end{tabular}
\end{table}

\subsubsection{Dataset-3}
\label{subsubsec:Dataset-3}

Dataset-3: \textit{Homicide Reports}(1980 - 2014), berisi mengenai data pembunuhan yang terjadi di negara US yang diambil dari FBI. Data kali ini diperoleh dari \url{https://www.kaggle.com/murderaccountability/homicide-reports} pada tanggal 29 April 2017(Contoh dataset telah dilampirkan pada lampiran~\ref{lamp:E-Contoh Datase1-3}). Data ini memiliki 24 atribut. Dalam eksperimen kali ini, akan dipilih satu atribut sebagai atribut kelas yaitu atribut \textit{Crime Solved} dan 6 atribut sebagai atribut prediktor yaitu: (1)\textit{Crime Type}; (2)\textit{Victim Age}; (3)\textit{Perpetrator Age}; (4)\textit{Perpetrator Ethnicity}; (5)\textit{Relationship}; (6)\textit{Weapon} . Atribut kelas tersebut nantinya diharapkan dapat melakukan prediksi untuk mengetahui apakah kejahatan bisa terpecahkan jika diberikan beberapa keterangan mengenai korban dan pelaku serta hubungan dan senjata yang digunakan untuk membunuh korban. Berikut merupakan penjelasan tiap atribut yang diperoleh dari \url{https://www.kaggle.com/murderaccountability/homicide-reports}:
\begin{enumerate}
	\item \texttt{Record ID: Numeric}
	\item \texttt{Agency Code: String}
	\item \texttt{Agency Name: String}
	\item \texttt{Agency Type: String}
	\item \texttt{City: String}
	\item \texttt{State: String}
	\item \texttt{Year: Numeric}
	\item \texttt{Month: String}
	\item \texttt{Incident: Numeric}
	\item \texttt{Crime Type: String} \textcolor{red}{*}\texttt{atribut prediktor}
	\item \texttt{Crime Solved: String} \textcolor{red}{*}\texttt{atribut kelas}
	\item \texttt{Victim Sex: String}
	\item \texttt{Victim Age: Numeric} \textcolor{red}{*}\texttt{atribut prediktor}
	\item \texttt{Victim Race: String}
	\item \texttt{Victim Ethnicity: String}
	\item \texttt{Perpetrator Sex: String}
	\item \texttt{Perpetrator Age: Numeric} \textcolor{red}{*}\texttt{atribut prediktor}
	\item \texttt{Perpetrator Race: String}
	\item \texttt{Perpetrator Ethnicity: String} \textcolor{red}{*}\texttt{atribut prediktor}
	\item \texttt{Relationship: String} \textcolor{red}{*}\texttt{atribut prediktor}
	\item \texttt{Weapon: String} \textcolor{red}{*}\texttt{atribut prediktor}
	\item \texttt{Victim Count: Numeric}
	\item \texttt{Perpetrator Count: Numeric}
	\item \texttt{Record Source: String}
\end{enumerate}

Dataset-3 asli ini memiliki ukuran sebesar 100MB. Pada eksperimen kali ini dilakukan manipulasi terhadap dataset-3 agar menjadi lebih besar. Manipulasi data dilakukan dengan cara melakukan pengulangan terhadap data yang sama hingga mencapai ukuran 335MB. Hal ini dilakukan agar data yang diumpankan kepada program mapreduce tidak terlalu kecil. Pada dataset-3 yang berukuran 335MB, eksperimen dilakukan dengan presentase data training dan data testing sebesar 70 : 30.

Data yang sudah diperbanyak tersebut dimasukkan ke dalam HDFS melalui perangkat lunak pada modul input seperti pada gambar~\ref{fig:Memasukkan dataset-3 ke dalam HDFS}

\begin{figure}[H]
	\centering
	\includegraphics[scale=0.65]{Eksperimen_Model/dataset3_0}
	\caption[Memasukkan dataset-3 ke dalam HDFS]{Memasukkan dataset-3 ke dalam HDFS}
	\label{fig:Memasukkan dataset-3 ke dalam HDFS}
\end{figure}

Setelah melakukan eksekusi program untuk melakukan pembuatan model dengan menggunakan 70\% dari dataset-2, diperoleh hasil model NBC yang tertera pada lampiran~\ref{lamp:E-Hasil Pembuatan Model Dataset-3}.

Setelah model NBC dihasilkan, dilakukan juga eksperimen testing untuk mengetahui kualitas dari NBC yang dihasilkan dengan menggunakan 30\% dari dataset-3, diperoleh hasil testing yang tertera pada lampiran~\ref{lamp:E-Hasil Testing Dataset-3}. Pada lampiran berikut dapat dilihat bahwa model NBC memiliki nilai \textit{accuracy} yang cukup tinggi. Hal tersebut dapat dilihat pada hasil \textit{matrix confusion} dan hasil evaluasi dari model NBC yang tertera pada Tabel~\ref{tab:confusion matrix - homicide} dan Tabel~\ref{tab:model evaluation - homicide}. \textit{Accuracy} memiliki nilai sebesar 96\%. Untuk nilai evaluasi model yang spesifik terhadap nilai kelas seperti nilai \textit{precision}, \textit{recall}, dan \textit{f-measure} juga memiliki nilai yang berkisar diantara 90\%. Dapat disimpulkan bahwa model NBC yang dibuat cukup bagus, karena nilai pehitungan evaluasi model NBC yang dihasilkan tidak ada yang berada dibawah 90\%.

\begin{table}[H]
\label{tab:confusion matrix - homicide}
\centering
\caption{\textit{Confusion Matrix - homicide reports}}
\begin{tabular}{ | c | c | c | }
\hline
& \textbf{no} & \textbf{yes} \\ \hline \hline
\textbf{no} & 165350 & 2796 \\ \hline
\textbf{yes} & 14377 & 372866 \\ \hline
\end{tabular}
\end{table}

\begin{table}[H]
\label{tab:model evaluation - homicide}
\centering
\caption{Hasil Evaluasi Model NBC - \textit{homicide reports}}
\begin{tabular}{ | c | c | c | c | c | }
\hline
\textit{\textbf{Class}} & \textit{\textbf{Accuracy}} & \textit{\textbf{Precision}} & \textit{\textbf{Recall}} & \textit{\textbf{F-Measure}}\\ \hline \hline
\textbf{no} & 0.969 & 0.920 & 0.983 & 0.922 \\ \hline
\textbf{yes} & 0.969 & 0.992 & 0.962 & 0.991 \\ \hline
\end{tabular}
\end{table}

\subsection{Performansi Big Data}
Eksperimen perangkat lunak yang dilakukan dengan menggunakan \textit{big data} memiliki tujuan untuk menguji kemampuan dari perangkat lunak terhadap sebuah \textit{big data} dan variabel-variabel pada Hadoop. Eksperimen ini dilakukan dengan menggunakan dataset "Homicide Reports" dari eksperimen pembuatan model dataset-3 pada bagian~\ref{subsubsec:Dataset-3}. Data tersebut diletakan pada HDFS tepatnya pada direktori \textit{input}. Terdapat dua buah variabel yang akan dibandingkan pada eksperimen ini, yaitu (1) ukuran blok HDFS, (2) jumlah ukuran data dan atribut prediktor yang digunakan. Eksperimen akan dilakukan sebanyak 3 kali untuk masing - masing variabel dengan nilai yang berbeda - beda.

Untuk setiap eksperimen, hanya akan dilakukan 1 kali eksekusi program berbasis \textit{MapReduce}, yaitu untuk pembuatan model NBC(pada modul \textit{train naive bayes}). Hal ini dikarenakan oleh data masukan pada program testing bergantung kepada data keluaran pada program training yang ukurannya relatif kecil, sehingga tidak bisa diubah dengan sembarangan. Setiap eksekusi akan melakukan 3 kali \textit{run} untuk ditentukan rata - rata dari waktu eksekusi program dan standar deviasi untuk setiap \textit{run}.

\subsubsection{Uji Pengaruh Ukuran Blok Terhadap Kecepatan}

Pengujian pengaruh ukuran blok terhadap waktu dilakukan dengan spesifikasi seperti berikut:

\begin{itemize}
	\item Jumlah \textit{slave node}: 4 \textit{node}
	\item Ukuran data: 1,11GB
	\item Jumlah atribut prediktor: 6 (1 bertipe numerik; 5 bertipe diskrit)
\end{itemize}

\begin{table}[H]
\label{tab:uji pengaruh ukuran blok}
\centering
\caption{Hasil uji pengaruh ukuran blok terhadap kecepatan program dalam detik}
\begin{tabular}{ | l | l | l | l | }
\hline
Ukuran blok HDFS & 32MB & 64MB & 128MB \\ \hline \hline
Run 1: Training & 870 & 610 & 1619 \\ \hline
Run 2: Training & 953 & 1670 & 1019 \\ \hline
Run 3: Training & 797 & 1670 & 1394 \\ \hline
\textbf{Rata - rata: Training} & \textbf{873} & \textbf{1318} & \textbf{1344} \\ \hline
\textbf{Standar deviasi: Training} & 78.053 & 611.991 & 303.108 \\ \hline
\end{tabular}
\end{table}

Hasil pengujian tertera pada Tabel~\ref{tab:uji pengaruh ukuran blok}. Pada Gambar~\ref{fig:Grafik pengaruh ukuran blok terhadap rata-rata waktu eksekusi} memperlihatkan grafik rata-rata waktu ekskusi. Dari grafik dapat dilihat bahwa ukuran blok memiliki pengaruh terhadap waktu eksekusi program. Untuk ukuran data sebesar 1.11GB, waktu eksekusi tercepat adalah pada ukuran blok 32 MB.

\begin{figure}[H]
	\centering
	\includegraphics[scale=0.74]{Eksperimen_Bigdata/ukuran_blok}
	\caption[Grafik pengaruh ukuran blok terhadap rata-rata waktu eksekusi]{Grafik pengaruh ukuran blok terhadap rata-rata waktu eksekusi}
	\label{fig:Grafik pengaruh ukuran blok terhadap rata-rata waktu eksekusi}
\end{figure}


\subsubsection{Uji Pengaruh Ukuran Data dan Jumlah Atribut Prediktor}

Pengujian pengaruh ukuran data dan jumlah atribut prediktor yang digunakan terhadap waktu dilakukan dengan spesifikasi seperti berikut:

\begin{itemize}
	\item Jumlah \textit{slave node}: 4 \textit{node}
	\item Ukuran blok: 32MB
\end{itemize}

\paragraph{Untuk Atribut Prediktor Sebanyak 6(numerik=1; diskrit=5) dan Atribut Kelas Sebanyak 1}

\begin{table}[H]
\label{tab:uji pengaruh ukuran data atr 6}
\centering
\caption{Hasil uji pengaruh ukuran data terhadap kecepatan program dengan atribut prediktor sebanyak 6 dan atribut kelas sebanyak 1 dalam detik}
\begin{tabular}{ | l | l | l | l | l | l | }
\hline
Ukuran data & 227,24MB & 454,47MB & 681,71MB & 908,95MB & 1,11GB \\ \hline \hline
Run 1: Training & 52 & 99 & 308 & 322 & 708 \\ \hline
Run 2: Training & 86 & 90 & 265 & 331 & 495 \\ \hline
Run 3: Training & 64 & 66 & 94 & 420 & 344 \\ \hline
\textbf{Rata - rata: Training} & \textbf{67.33} & \textbf{85} & \textbf{222.33} & \textbf{357.66} & \textbf{515.66} \\ \hline
\textbf{Standar deviasi: Training} & \textbf{17.24} & \textbf{17.05} & \textbf{113.2} & \textbf{54.16} & \textbf{182.87} \\ \hline
\end{tabular}
\end{table}

Hasil pengujian tertera pada Tabel~\ref{tab:uji pengaruh ukuran data atr 6}. Pada Gambar~\ref{fig:Grafik pengaruh ukuran data terhadap rata-rata waktu eksekusi atr 6} memperlihatkan grafik rata-rata waktu ekskusi. Pada grafik dapat dilihat bahwa pertambahan nilai eksekusi waktu sebanding dengan pertambahan ukuran data dan kenaikannya terjadi secara linear.


Khusus untuk atribut prediktor sebanyak 6 dengan yang memiliki tipe diskrit sebanyak 5 dan yang memiliki tipe numerik sebanyak 1, didapat bahwa maksimum ukuran data yang bisa dimasukkan hanya sebanyak 1.11GB saja. Hal tersebut terjadi dikarenakan oleh perhitungan khusus untuk atribut numerik yang melakukan pengulangan sebanyak 2 kali. Pengulangan sebanyak 2 kali dilakukan pada perhitungan standard deviasi atribut numerik untuk menghitung nilai rata - rata yang nantinya digunakan untuk menghitung nilai standard deviasi. Hal ini mengakibatkan node pekerja yang melakukan pekerjaan tersebut perlu menyimpan seluruh data pada memori ram milik \textit{node} tersebut. Sehingga mengakibatkan terjadinya error pada mesin JVM karena kekurangan memori (\textit{java heap space error}). Perhitungan tersebut dilakukan pada program saat menjalankan fase \textit{reduce}

\begin{figure}[H]
	\centering
	\includegraphics[scale=0.8]{Eksperimen_Bigdata/ukuran_data_6}
	\caption[Grafik pengaruh ukuran data terhadap rata-rata waktu eksekusi dengan atribut prediktor sebanyak 6]{Grafik pengaruh ukuran data terhadap rata-rata waktu eksekusi dengan atribut prediktor sebanyak 6}
	\label{fig:Grafik pengaruh ukuran data terhadap rata-rata waktu eksekusi atr 6}
\end{figure}

\paragraph{Untuk Atribut Prediktor Sebanyak 4(diskrit=4) dan Atribut Kelas Sebanyak 1}

\begin{table}[H]
\label{tab:uji pengaruh ukuran data atr 4}
\centering
\caption{Hasil uji pengaruh ukuran data terhadap kecepatan program dengan atribut prediktor sebanyak 4 dan atribut kelas sebanyak 1 dalam detik}
\begin{tabular}{ | l | l | l | l | l | l | l | }
\hline
Ukuran data & 500MB & 1GB & 2GB & 4GB & 10GB & 20GB \\ \hline \hline
Run 1: Training & 46 & 58 & 87 & 150 & 341 & 679  \\ \hline
Run 2: Training & 41 & 65 & 90 & 159 & 339 & 706  \\ \hline
Run 3: Training & 42 & 60 & 87 & 159 & 342 & 685  \\ \hline
\textbf{Rata - rata: Training} & \textbf{43} & \textbf{61} & \textbf{88} & \textbf{156} & \textbf{340.66} & \textbf{690} \\ \hline
\textbf{Standar deviasi: Training} & \textbf{2.64} & \textbf{3.6} & \textbf{1.73} & \textbf{5.19} & \textbf{1.52} & \textbf{14.17} \\ \hline
\end{tabular}
\end{table}

Hasil pengujian tertera pada Tabel~\ref{tab:uji pengaruh ukuran data atr 4}. Pada Gambar~\ref{fig:Grafik pengaruh ukuran data terhadap rata-rata waktu eksekusi atr 4} memperlihatkan grafik rata-rata waktu ekskusi. Pada grafik dapat dilihat bahwa pertambahan nilai eksekusi waktu sebanding dengan pertambahan ukuran data dan kenaikannya terjadi secara linear.

\begin{figure}[H]
	\centering
	\includegraphics[scale=0.8]{Eksperimen_Bigdata/ukuran_data_4}
	\caption[Grafik pengaruh ukuran data terhadap rata-rata waktu eksekusi dengan atribut prediktor sebanyak 4]{Grafik pengaruh ukuran data terhadap rata-rata waktu eksekusi dengan atribut prediktor sebanyak 4}
	\label{fig:Grafik pengaruh ukuran data terhadap rata-rata waktu eksekusi atr 4}
\end{figure}

\paragraph{Untuk Atribut Prediktor = 2(diskrit = 2) dan Atribut Kelas Sebanyak 1}

\begin{table}[H]
\label{tab:uji pengaruh ukuran data atr 2}
\centering
\caption{Hasil uji pengaruh ukuran data terhadap kecepatan program dengan atribut prediktor sebanyak 2 dan atribut kelas sebanyak 1 dalam detik}
\begin{tabular}{ | l | l | l | l | l | l | l | }
\hline
Ukuran data & 500MB & 1GB & 2GB & 4GB & 10GB & 20GB \\ \hline \hline
Run 1: Training & 37 & 50 & 78 & 138 & 339 & 597  \\ \hline
Run 2: Training & 38 & 52 & 80 & 137 & 337 & 601  \\ \hline
Run 3: Training & 37 & 53 & 77 & 140 & 334 & 599  \\ \hline
\textbf{Rata - rata: Training} & \textbf{37.33} & \textbf{51.76} & \textbf{78.33} & \textbf{138.33} & \textbf{336.66} & \textbf{599} \\ \hline
\textbf{Standar deviasi: Training} & \textbf{0.57} & \textbf{1.75} & \textbf{1.52} & \textbf{1.52} & \textbf{2.51} & \textbf{2} \\ \hline
\end{tabular}
\end{table}

Hasil pengujian tertera pada Tabel~\ref{tab:uji pengaruh ukuran data atr 2}. Pada Gambar~\ref{fig:Grafik pengaruh ukuran data terhadap rata-rata waktu eksekusi atr 2} memperlihatkan grafik rata-rata waktu ekskusi. Pada grafik dapat dilihat bahwa pertambahan nilai eksekusi waktu sebanding dengan pertambahan ukuran data dan kenaikannya terjadi secara linear.

\begin{figure}[H]
	\centering
	\includegraphics[scale=0.74]{Eksperimen_Bigdata/ukuran_data_2}
	\caption[Grafik pengaruh ukuran data terhadap rata-rata waktu eksekusi dengan atribut prediktor sebanyak 2]{Grafik pengaruh ukuran data terhadap rata-rata waktu eksekusi dengan atribut prediktor sebanyak 2}
	\label{fig:Grafik pengaruh ukuran data terhadap rata-rata waktu eksekusi atr 2}
\end{figure}

\subsubsection{Kesimpulan}
Berdasarkan eksperimen - eksperimen yang telah dilakukan, maka dapat diambil kesimpulan sebagai berikut:

\begin{enumerate}[label=(\alph*)]
	\item Ukuran blok berpengaruh pada performa kecepatan eksekusi dari program yang dijalankan berdasarkan besaran ukuran data yang diproses. Oleh karena itu, pemilihan ukuran blok yang tepat perlu dilakukan dengan baik, agar program yang dieksekusi dapat berjalan secara cepat.
	\item Ukuran data berpengaruh pada kecepatan eksekusi program. Tetapi untuk beberapa kasus perlu diperhatikan ukuran data dengan kapasitas \textit{hardware} yang digunakan, karena jika kapasitas \textit{hardware} yang digunakan tidak mampu menangani jumlah data maka akan terjadi \textit{error} pada program.
	\item Untuk variabel jumlah atribut prediktor, dapat dibuktikan bahwa hal tersebut dapat mempengaruhi waktu eksekusi program. Hal ini dapat dilhat pada gambar~\ref{fig:Grafik pengaruh ukuran data terhadap rata-rata waktu eksekusi mix}. Dapat dilihat bahwa jumlah atribut prediktor memiliki pengaruh terhadap waktu eksekusi.	
	%Marking one	
	\item Perbedaan waktu eksekusi data yg memiliki atribut numerik dibandingkan hanya memiliki atribut diskret memiliki perbedaan yang sangat signifikan. Seperti pada gambar~\ref{fig:Grafik pengaruh ukuran data terhadap rata-rata waktu eksekusi mix}, untuk data yang memiliki atribut numerik memiliki waktu eksekusi yang jauh lebih lama dibandingkan jika seluruh datanya hanya berisi atribut diskrit. Hal tersebut disebabkan karena perhitungan komputasi yang melibatkan atribut numerik memiliki kompleksitas \verb|O(2n)| yang menyebabkan waktu eksekusi akan 2 kali lipat lebih lama daripada perhitungan pada atrbut diskrit. Selain itu, akan melakukan penulisan ke dalam HDFS (untuk output) untuk setiap iterasi pada atribut numerik yang menyebabkan grafik pada gambar~\ref{fig:Grafik pengaruh ukuran data terhadap rata-rata waktu eksekusi mix} memiliki perbedaan yang sangat jauh (proses penulisan ke dalam HDFS untuk output juga membutuhkan waktu lebih banyak).
	\begin{figure}[H]
	\centering
	\includegraphics[scale=0.74]{Eksperimen_Bigdata/ukuran_data-mix_ALL}
	\caption[Grafik pengaruh ukuran data terhadap rata-rata waktu eksekusi dengan jumlah dan tipe atribut prediktor berbeda]{Grafik pengaruh ukuran data terhadap rata-rata waktu eksekusi dengan jumlah dan tipe atribut prediktor berbeda}
	\label{fig:Grafik pengaruh ukuran data terhadap rata-rata waktu eksekusi mix}
	\end{figure}
	%Marking two
	\item Perbedaan ukuran data yang dapat ditangani oleh perangkat lunak dengan spesifikasi yang sudah dijelaskan jika semua atribut berupa diskret dengan yang memiliki atribut numerik terlihat sangat jauh berbeda seperti pada gambar~\ref{fig:Grafik pengaruh ukuran data terhadap rata-rata waktu eksekusi mix}. Untuk data yang memiliki atribut numerik hanya mampu memproses hingga ukurannya mencapai 1,11GB saja, selebihnya akan terkena limit pada memori \textit{node} pekerja yang terbatas. Hal tersbut terjadi karena proses penanganan atribut numerik perlu menyimpan objek - objek bertipe \texttt{double} sebanyak 2 kali lipat yang membutuhkan memori cukup besar untuk tiap objeknya (untuk objek bertipe \texttt{double}, java akan mengalokasikan memori sebesar 8 bytes. Sumber: \url{http://www.javamex.com/tutorials/memory/object_memory_usage.shtml}).
	\item Untuk atribut bertipe numerik program belum mampu secara cepat melakukan perhitungan standard deviasi. Karena, perlu melakukan iterasi sebanyak dua kali, yaitu untuk menghitung nilai rata - rata, lalu menghitung nilai standard deviasi. Perhitungan tersebut dilakukan pada \textit{node} \textit{reducer} yang menggunakan memori pada JVM, dimana JVM juga menggunakan memori pada RAM komputer milik \textit{node reducer}.
	
	
\end{enumerate}