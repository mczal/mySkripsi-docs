\chapter{Pendahuluan}
% By Mczal \label{chap:intro}

\section{Latar Belakang}
% By Mczal \label{sec:motivation}
% \textsl{
Seiring dengan berkembangnya teknologi, kebutuhan akan penggunaan internet melaju sangat pesat menjadi sangat tinggi. Tidak dapat disangkal, bahwa kebanyakan dari manusia masa kini sudah melihat internet sebagai kebutuhan primer-nya, karena kemudahan dan manfaat yang ditawarkan oleh internet yang sangat banyak. Hal tersebut menyebabkan frekuensi penggunaan internet yang semakin tinggi pula, sehingga secara tidak langsung internet telah menjadi sarana utama dalam mendapatkan informasi dan telah berhasil meniadakan batasan informasi yang bisa diakses dari mana saja.

Dengan begitu, banyak perusahaan yang menjadikan internet sebagai salah satu sarana utama untuk mengembangkan produk/jasa yang mereka miliki, melihat pengguna-nya yang sangat banyak dan bermacam - macam dari segala penjuru dunia yang juga menggunakan internet. Saking banyaknya pengguna dan data yang dioper di dalam internet maka kebutuhan untuk mengolah data yang sangat bervariasi dan jumlah yang sangat besar dengan kecepatan yang tinggi menjadi pokok permasalahan yang dihadapi saat ini \textit{( Big Data)}.

% def, contoh, asal -> Big Data
{\it Big Data} merupakan suatu terminologi modern untuk sekumpulan data yang memiliki kesulitan tersendiri untuk diproses dengan cara tradisional (menggunakan satu buah komputer).
3 hal terpenting yang menjadi pokok permasalahan dalam {\it Big Data} adalah : (1) mengolah data yang berjumlah sangat besar, (2) mengolah data yang memiliki tipe sangat bermacam - macam / variatif, (3) mengolah data dengan performa yang optimal. {\it Big Data} tidak melulu berasal dari internet, di dalam kehidupan kita sehari - hari sering kali kita berurusan dengan data, seperti data pada sensor sidik jari ketika absensi, data pembelian pada supermarket, data sensor kelembaban udara pada 10 tahun terakhir untuk memprediksi cuaca, kenaikan dan penurunan harga saham, {\it bitcoin}, dsb. {\it Big Data} menjadi topik yang diminati karena dengan data yang begitu banyak, dapat diteliti pola yang terjadi pada data tersebut selama beberapa kurun waktu tertentu untuk digunakan dalam menganalisis data dan membuat keputusan serta memberikan prediksi kemunculan data berikutnya dengan tingkat akurasi yang tinggi berdasarkan data yang dipelajari. Perusahaan - perusahaan saat ini tengah memulai untuk mengumpulkan setiap data yang dapat mereka peroleh dari {\it customer} untuk melihat pola aktifitas {\it customer} mereka dan membuat keputusan yang dapat menguntungkan perusahaan berdasarkan hal tersebut. Tentu saja hal ini tidak dapat dilakukan menggunakan teknik komputasi yang tradisional (menggunakan satu buah komputer berteknologi tinggi), karena biaya dan waktu yang terlalu mahal dan lama.

% def, for -> Hadoop
\textit{Apache Hadoop} merupakan platform yang dibuat untuk menangani permasalahan yang muncul pada \textit{Big Data} dan melakukan analisis pada \textit{Big Data}. Hadoop merupakan sebuah framework \textit{open-source} yang terdiri dari beberapa \textit{cluster} yang saling bekerja sama untuk mengolah data berdasarkan sistem yang terdistribusi dan mampu melibatkan ratusan bahkan ribuan \textit{cluster} yang dapat menjadi node \textit{worker}-nya. Hadoop memiliki dua komponen utama yaitu Hadoop Distributed File System (HDFS) dan MapReduce. Map Reduce adalah sebuah model fungsi pemrograman untuk memproses data yang sangat besar.
\textit{Map-Reduce} menggunakan algoritma paralel dan terdistribusi. Fungsi \textit{Map-Reduce} tersebut akan menyaring, memperkecil, dan melakukan agregasi terhadap data sehingga data yang tidak diperlukan akan dihilangkan. Hadoop Distributed File System (HDFS) adalah sebuah sistem file yang terdistribusi yang didesain untuk beroperasi di dalam suatu kumpulan hardware \textit{(a set of commodity hardware)}. Jika dibandingkan dengan \textit{file-system} lainnya, Hadoop Distributed File System dirancang untuk menyimpan data set yang besar dan memiliki bandwidth yang tinggi untuk melakukan streaming data tersebut.
%Map Reduce terdiri dari 2 fungsi, yaitu fungsi \textit{Map} yang dapat menyaring dan mengurutkan data dan fungsi \textit{Reduce} yang dapat melakukan perhitungan atau operasi pengurangan data. 


%def, problem -> Naive Bayes
Di samping itu, Data Mining merupakan teknologi baru yang sangat berguna untuk membantu perusahaan-perusahaan menemukan informasi yang sangat penting dari gudang data mereka. Melihat perkembangan dan pertumbuhan data yang kian semakin tinggi, teknik data mining sangat cocok untuk diimplementasikan pada \textit{Big Data}. Karena, diharapkan teknik data mining memiliki tingkat akurasi yang tinggi sebanding dengan volume data yang kian meninggi. Kakas \textit{data mining} meramalkan tren dan sifat-sifat perilaku bisnis yang sangat berguna untuk mendukung pengambilan keputusan penting. \textit{Data Mining} dapat menjawab pertanyaan-pertanyaan bisnis yang dengan cara tradisional memerlukan banyak waktu untuk menjawabnya. \textit{Data Mining} mengeksplorasi data - data yang sudah ada untuk menemukan pola - pola yang tersembunyi dan mencari informasi pemrediksi yang mungkin saja terlupakan oleh para pelaku bisnis karena tidak terpikirkan sebelumnya oleh mereka.

Fokus penelitian tugas akhir ini adalah untuk menggunakan sistem terdistribusi hadoop dalam memecahkan 3 masalah utama yang dimiliki oleh \textit{Big Data} dalam menerapkan algoritma teknik data mining \textit{(Naive Bayes Classifier)} dalam melakukan klasifikasi berdasarkan data yang diberikan.
%}

\section{Rumusan Masalah}
Dari latar belakang tersebut, rumusan masalah yang dibahas pada penelitian ini adalah :
\begin{enumerate}
	\item Bagaimana merancang algoritma {\it Naive Bayes Classifier} berbasis \textit{MapReduce} pada lingkungan sistem terdistribusi Hadoop ?
	\item Bagaimana mengimplementasikan algoritma {\it Naive Bayes Classifier} berbasis \textit{MapReduce} pada lingkungan sistem terdistribusi Hadoop ?
	\item Bagaimana melakukan pengujian pada algoritma \textit{Naive Bayes Classifier} berbsis \textit{MapReduce} ?	
	\item Bagaimana melakukan eksperimen terhadap algoritma \textit{Naive Bayes Classifier} pada lingkungan terdistribusi Hadoop menggunakan \textit{Big Data}?
\end{enumerate}

\section{Tujuan}
%Penelitian ini bertujuan untuk membangun perangkat lunak yang dapat melakukan klasifikasi menggunakan algoritma Naive Bayes dengan memanfaatkan MapReduce di lingkungan sistem terdistribusi Hadoop.
%Perangkat lunak ini nantinya akan bisa melakukan klasifikasi untuk \textit{Big Data} yang dapat berguna dalam mempelajari pola dari \textit{Big Data}/ input itu sendiri \textit{(supervised learning)}.
Berdasarkan identifikasi masalah, tujuan penelitian sebagai berikut:
\begin{enumerate}
	\item Merancang algoritma \textit{Naive Bayes Classifier} berbasis \textit{MapReduce} pada lingkungan terdistribusi Hadoop.
	\item Mengimplementasikan algoritma \textit{Naive Bayes Classifier} berbasis \textit{MapReduce} pada lingkungan terdistribusi Hadoop.
	\item Menguji hasil implementasi algoritma \textit{Naive Bayes Classifier} untuk analisis \textit{Big Data}.
	\item Melakukan eksperimen pada algoritma \textit{Naive Bayes Classifier} pada lingkungan terdistribusi Hadoop menggunakan Big Data
\end{enumerate}

%\section{Batasan Masalah}

%Batasan masalah dari penelitian ini antara lain :
%\begin{enumerate}
%	\item abc
%	\item def
%\end{enumerate}

\section{Metodologi Penelitian}

Langkah-langkah yang dilakukan dalam penelitian ini adalah :
\begin{enumerate}
	%\item Studi Literatur \\
	%Tahap ini dilakukan untuk mengetahui apa itu Sistem Terdistribusi Hadoop dan mengapa menggunakan Sistem Terdistribusi Hadoop dapat menguntungkan untuk mengolah \textit{Big data}, lalu untuk mempelajari teknik klasifikasi dengan menggunakan algoritma Naive Bayes pada Sistem Terdistribusi Hadoop.
	%\item Analisis Email \\
	%Tahap ini dilakukan untuk mempersiapkan email yang menjadi input dari program dan mengolah data input \textit{(data pre-processing)} dengan melihat variasi dari atribut yang dimiliki oleh email pada umumnya sebelum dimasukan kedalam program.
	%\item Pengumpulan Data \\
	%Tahap ini dilakukan untuk mengumpulkan data email dari berbagai sumber dengan jumlah yang sangat banyak untuk melakukan klasifikasi.
	%\item Perancangan Perangkat Lunak \\
	%Tahap ini dilakukan untuk merancang perangkat lunak agar dapat mengolah data input dan menerapkan algoritma Naive Bayes pada Sistem Terdistribusi Hadoop.
	%\item Implementasi Perangkat Lunak \\
	%Tahap ini dilakukan untuk mengimplementasikan perangkat lunak untuk mengklasifikasi email spam dan non-spam.
	%\item Pengujian Perangkat Lunak \\
	%Tahap ini dilakukan untuk menguji apakah perangkat lunak dapat melakukan klasifikasi pada email spam/non-spam dengan tepat (tingkat error yang dapat dimaklumi).
	%\item Penarikan Kesimpulan \\
	%Tahap ini akan menentukan dari hasil penelitian yang telah dilakukan, apakah semua masalah pada rumusan masalah dapat terselesaikan atau tidak.
	\item Melakukan studi literatur tentang sistem terdistribusi Hadoop dan tools lainnya yang dapat membantu
		\item Melakukan studi literatur tentang klasifikasi menggunakan algoritma \textit{naive bayes}
		\item Mempelajari Hadoop MapReduce dan membuat program - program kecil yang dapat mendukung implementasi dari algoritma klasifikasi \textit{naive bayes} berbasis MapReduce
		\item Merancang algoritma \textit{Naive Bayes Classifier} berbasis \textit{MapReduce} 
		\item Mengumpulkan data yang dianalisis/diuji (input)
		%\item Merancang teknik {\it preprocessing data} yang diperlukan
		\item Melakukan implementasi klasifikasi menggunakan algoritma \textit{naive bayes classifier} pada sistem terdistribusi Hadoop dengan \textit{Big Data}
		\item Menganalisis studi kasus untuk data yang berukuran kecil, menengah, dan sangat besar \textit{(Big Data)}
		\item Merancang teknik analisis hasil data dari output pada {\it Hadoop M-R Job}
		\item Melakukan pengujian (dan eksperimen) untuk menguji performa sistem
		%\item Menulis dokumen skripsi
\end{enumerate}


\section{Sistematika Pembahasan}

Sistematika pembahasan penelitian ini, yaitu :

\begin{enumerate}
	\item Bab 1 Pendahuluan, berisi tentang permasalahan utama yang dibahas pada penelitian ini dan dipecahkan menjadi beberapa poin penting, tujuan dari penelitian, batasan masalah, metodologi penelitian yang digunakan, dan sistematika pembahasan pada penelitian ini. 
	\item Bab 2 Landasan Teori, berisi tentang teori dasar dan pengetahuan mengenai Sistem Terdistribusi Hadoop dan Algoritma \textit{Naive Bayes Classifier}. Pada bab ini dijelaskan juga mengenai beberapa \textit{framework} yang digunakan dalam membangun perangkat lunak.
	\item Bab 3 Analisis, berisi tentang analisis masalah yang telah dideskripsikan pada Bab 1 dan menentukan beberapa kebutuhan dari perangkat lunak. Selain itu juga melakukan perhitungan manual algoritma \textit{naive bayes classification} dengan studi kasus.
	\item Bab 4 Perancangan Perangkat Lunak, berisi tentang rancangan perangkat lunak yang dibangun. Perancangan perangkat lunak meliputi perancangan antarmuka, diagram kelas lengkap, dan rincian metode - metode yang ada pada kelas.
	\item Bab 5 Implementasi, Pengujian, dan Eksperimen Perangkat Lunak, berisi tentang hasil dari implementasi, pengujian, dan eksperimen yang dilakukan pada perangkat lunak pada lingkungan terdistribusi Hadoop.
	\item Bab 6 Kesimpulan dan Saran, berisi tentang kesimpulan atas hasil penelitian yang telah dilakukan, apakah semua masalah pada rumusan masalah dapat terselesaikan atau tidak, serta saran untuk penelitian yang masih bisa dikembangkan dari penelitian ini.
\end{enumerate}
