\chapter{Perancangan}
Berdasarkan analisis yang telah dilakukan, terdapat beberapa hal yang perlu dirancang untuk pembangunan perangkat lunak naive bayes berbasis \textit{hadoop mapreduce}. Pada bab ini akan dijelaskan perancangan yang diperlukan untuk membangun perangkat lunak yaitu perancang-
an antarmuka, diagram kelas rinci, serta rincian metode.
			
\section{Perancangan Antarmuka}

Perangkat lunak \textit{naive bayes classification} memiliki 6 buah tampilan untuk yang tidak berbasis \textit{MapReduce}, yaitu: (1) \textit{Dashboard} (2) \textit{Input Set Manager} (3) \textit{Renew Model Manager} (4) \textit{Testing Manager} (5) \textit{Classification Manager} (6) \textit{Error Rate Dashboard}. Untuk program yang berbasis \textit{MapReduce} tidak akan memiliki antarmuka yang khusus, karena program hanya perlu dijalankan dengan menggunakan CLI (\textit{command line interface}). Berikut adalah penjelasan dan gambar dari tiap antarmuka yang dirancang:

\subsection{\textit{Dashboard}}

\begin{figure}[H]
	\centering
	\includegraphics[scale=0.45]{Mockup/mockup_dashboard_0}
	\caption[input-set-gui-1]{Dashboard}
	\label{fig:input-set-gui-1}
\end{figure}
\textit{Dashboard} dibuat untuk memudahkan user dalam memonitor model NBC yang telah dimasukkan ke dalam perangkat lunak yang dibangun. Berikut penjelasan lebih lanjut mengenai tiap komponen pada rancangan \textit{dashboard} yang dibuat:
\begin{enumerate}
	\item Berisi nama - nama atribut kelas dan total frekeunsi kemunculannya tiap nilai.
	\item Berisi nama - nama atribut prediktor dan frekuensi kemunculannya untuk prediktor bertipe diskrit dan \textit{mean} \verb|&| \textit{sigma} untuk yang bertipe numerik.
	\item \textit{Bayesian model} merupakan model dari NBC yang akan digunakan untuk testing dan klasifikasi. Model ini merupakan model yang langsung di-import dari hasil training di dalam HDFS.
\end{enumerate}

\subsection{\textit{Input Set Manager}}
\begin{figure}[H]
	\centering
	\includegraphics[scale=0.4]{Mockup/input-set-gui-1}
	\caption[\textit{Input Set Manager}]{\textit{Input Set Manager}}
	\label{fig:Input Set Manager}
\end{figure}
\textit{Input Set Manager} dibuat untuk memudahkan user melakukan input data ke dalam HDFS menggunakan perangkat lunak yang dibuat. Berikut penjelasan lebih lanjut mengenai tiap komponen pada rancangan \textit{Input Set Manager} yang dibuat:
\begin{enumerate}
	\item User dapat memilih tipe model input yang sudah ada dalam HDFS.
	\item Jika ingin membuat tipe model input baru pada HDFS, maka user perlu mengisi kolom ini dan mengisi nama model yang diinginkan.
	\item User dapat memilih file input yang akan dikirimkan ke dalam HDFS. User dapat memilih > 1 file sekaligus.
	\item User dapat memilih file info mengenai file input, yang dikirimkan ke dalam HDFS.
	\item User dapat memilih presentase pembagian data antara data \textit{training} dan data \textit{testing} dari keseluruhan data input yang akan dimasukkan ke dalam HDFS.
	\item Setelah memillih file info, user dapat memilih atribut mana saja yang akan digunakan untuk training. User juga dapat memilih tipe(diskrit/numerik) dari atribut tersebut beserta jenisnya (kelas/prediktor).
\end{enumerate}

\subsection{\textit{Renew Model Manager}}
\begin{figure}[H]
	\centering
	\includegraphics[scale=0.45]{Mockup/mockup_renewmodel_manager}
	\caption[\textit{Renew Model Manager}]{\textit{Renew Model Manager}}
	\label{fig:Renew Model Manager}
\end{figure}
\textit{Renew Model Manager} dibuat agar user selalu bisa memperbaharui model NBC pada perangkat lunak yang dibikin.
\begin{enumerate}
	\item User dapat memilih file model NBC hasil dari training dari sistem penyimpanan \textit{local}.
	\item User dapat memilih file model NBC hasil dari training langsung dari HDFS.
\end{enumerate}

\subsection{\textit{Testing Manager}}
\begin{figure}[H]
	\centering
	\includegraphics[scale=0.45]{Mockup/mockup_testing_manager}
	\caption[\textit{Testing Manager}]{\textit{Testing Manager}}
	\label{fig:Testing Manager}
\textit{Testing Manager} dibuat untuk melakukan testing pada model NBC yang sudah di-import ke dalam program sebelumnya.
\end{figure}
\begin{enumerate}
	\item User dapat memilih file input dan file info dari penyimpanan \textit{local} milik user.
	\item User dapat memilih file testing yang sudah ada di dalam HDFS dengan memilih model input direktori pada HDFS.
\end{enumerate}

\subsection{\textit{Classification Manager}}
\begin{figure}[H]
	\centering
	\includegraphics[scale=0.45]{Mockup/mockup_classification_manager}
	\caption[\textit{Classification Manager}]{\textit{Classification Manager}}
	\label{fig:Classification Manager}
\end{figure}
\textit{Classification Manager} dapat digunakan untuk mengklasifikasi satu record input/kasus yang secara langsung diisi sendiri oleh user yang menggunakannya terhadap model NBC yang sudah ada pada perangkat lunak sebelumnya.
\begin{enumerate}
	\item User memilih nilai prediktor untuk kasus baru (prediktor dapat berupa dropdown untuk yang bertipe diskrit dan \textit{number} untuk yang bertipe numerik)
	\item User dapat memilih kelas yang menjadi prediksi sebelumnya dari user untuk diperiksa kebenarannya jika menggunakan program setelah diklasifikasikan menggunakan model NBC yang sudah ada.
	\item Hasil dari klasifikasi yang telah dijalankan.
\end{enumerate}

\subsection{\textit{Error Rate Dashboard}}
\begin{figure}[H]
	\centering
	\includegraphics[scale=0.45]{Mockup/mockup_errorrate}
	\caption[\textit{Error Rate Dashboard}]{\textit{Error Rate Dashboard}}
	\label{fig:Error Rate Dashboard}
\end{figure}
\textit{Error Rate Dashboard} dibuat untuk memonitor hasil \textit{error rate} yang sudah dihitung setelah menjalani proses testing. 
\begin{enumerate}
	\item \textit{Confusion matrix} untuk setiap atribut kelas.
	\item Error rate yang akan dihasilkan setelah melakukan klasifikasi meliputi: (1)$Accuracy$; (2)$Precision$; (3)$Recall$; (4)$F-Measure$.
\end{enumerate}



\section{Diagram Kelas Lengkap}
Berikut adalah penjelasan dari kelas - kelas yang ada pada keempat modul yang dibuat dan  beserta penjelasan setiap atribut dan operasi yang dimiliki oleh kelas - kelas.

\subsection{Modul Kelola Input}


\subsection{Modul \textit{Train Naive Bayes M-R Based}}


\subsection{Modul \textit{Testing Naive Bayes M-R Based}}


\subsection{Modul Klasifikasi Naive Bayes}


