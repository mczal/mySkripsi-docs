%_____________________________________________________________________________
%=============================================================================
% data.tex v8 (02-10-2016) \ldots dibuat oleh Lionov - Informatika FTIS UNPAR
%
% Perubahan pada versi 8 (02-10-2016)
%	- Perubahan keterangan pada spacing: Otomatis spasi 1 untuk buku skripsi 
%	  final dan 1.5 untuk buku sidang
%	- Penggunaan kantlipsum
%_____________________________________________________________________________
%=============================================================================

%=============================================================================
% 								PETUNJUK
%=============================================================================
% Ini adalah file data (data.tex)
% Masukkan ke dalam file ini, data-data yang diperlukan oleh template ini
% Cara memasukkan data dijelaskan di setiap bagian
% Data yang WAJIB dan HARUS diisi dengan baik dan benar adalah SELURUHNYA !!
% Hilangkan tanda << dan >> jika anda menemukannya
%=============================================================================

%_____________________________________________________________________________
%=============================================================================
% 								BAGIAN 0
%=============================================================================
% PERHATIAN!! PERHATIAN!! Bagian ini hanya ada untuk sementara saja
% Jika "DAFTAR ISI" tidak bisa berada di bagian tengah halaman, isi dengan XXX
% jika sudah benar posisinya, biarkan kosong (i.e. \daftarIsiError{ })
%=============================================================================
\daftarIsiError{ }
%=============================================================================

%_____________________________________________________________________________
%=============================================================================
% 								BAGIAN I
%=============================================================================
% Tambahkan package2 lain yang anda butuhkan di sini
%=============================================================================
\usepackage{booktabs} 
\usepackage[table]{xcolor}
\usepackage{longtable}
\usepackage{amssymb}
\usepackage{todo}
\usepackage{verbatim} 		%multilne comment
\usepackage{pgfplots}

% START CUSTOM FAHRIZAL SEPTRIANTO
% START CUSTOM PACKAGE SECTION BY MOHAMAD FAHRIZAL S.
\usepackage{xcolor}
\usepackage{url}
\usepackage{multirow}

\usepackage{amsmath}
\usepackage[noend]{algpseudocode}
\usepackage{algorithm}

\usepackage{relsize}
\usepackage{listings}

% END CUSTOM PACKAGE SECTION BY MOHAMAD FAHRIZAL S.

%=============================================================================

%_____________________________________________________________________________
%=============================================================================
% 								BAGIAN II
%=============================================================================
% Mode dokumen: menetukan halaman depan dari dokumen, apakah harus mengandung 
% prakata/pernyataan/abstrak dll (termasuk daftar gambar/tabel/isi) ?
% - kosong : tidak ada halaman depan sama sekali (untuk dokumen yang 
%            dipergunakan pada proses bimbingan)
% - cover : cover saja tanpa daftar isi, gambar dan tabel
% - sidang : cover, daftar isi, gambar, tabel 
% - sidang_akhir : mode sidang + abstrak + abstract
% - final : seluruh halaman awal dokumen (untuk cetak final)
% Jika tidak ingin mencetak daftar tabel/gambar (misalkan karena tidak ada 
% isinya), edit manual di baris 439 dan 440 pada file main.tex
%=============================================================================
 \mode{kosong}
% \mode{cover}
% \mode{sidang}
% \mode{sidang_akhir}
%\mode{final} 
%=============================================================================

%_____________________________________________________________________________
%=============================================================================
% 								BAGIAN III
%=============================================================================
% Line numbering: penomoran setiap baris, otomatis di-reset setiap berganti
% halaman
% - yes: setiap baris diberi nomor
% - no : baris tidak diberi nomor, otomatis untuk mode final
%=============================================================================
\linenumber{no}
%=============================================================================

%_____________________________________________________________________________
%=============================================================================
% 								BAGIAN IV
%=============================================================================
% Linespacing: jarak antara baris 
% - single	: wajib (dan otomatis jika ingin mencetak buku skripsi, opsi yang 
%			  disediakan untuk bimbingan, jika pembimbing tidak keberatan 
%			  (untuk menghemat kertas)
% - onehalf	: default dan wajib (dan otomatis) jika ingin mencetak dokumen
%             untuk sidang.
% - double 	: jarak yang lebih lebar lagi, jika pembimbing berniat memberi 
%             catatan yg banyak di antara baris (dianjurkan untuk bimbingan)
%=============================================================================
\linespacing{single}
%\linespacing{onehalf}
%\linespacing{double}
%=============================================================================

%_____________________________________________________________________________
%=============================================================================
% 								BAGIAN V
%=============================================================================
% Tidak semua skripsi memuat gambar dan/atau tabel. Untuk skripsi yang seperti
% itu, tidak diperlukan Daftar Gambar dan Daftar Tabel. Sayangnya hal ini 
% sulit dilakukan secara manual karena membutuhkan kedisiplinan pengguna 
% template.  
% Jika tidak akan menampilkan Daftar Gambar/Tabel, isi dengan NO. Jika ingin
% menampilkan, kosongkan parameter (i.e. \gambar{ }, \tabel{ })
%=============================================================================
\gambar{ }
\tabel{ }
%=============================================================================

%_____________________________________________________________________________
%=============================================================================
% 								BAGIAN VI
%=============================================================================
% Bab yang akan dicetak: isi dengan angka 1,2,3 s.d 9, sehingga bisa digunakan
% untuk mencetak hanya 1 atau beberapa bab saja
% Jika lebih dari 1 bab, pisahkan dengan ',', bab akan dicetak terurut sesuai 
% urutan bab (e.g. \bab{1,2,3}).
% Untuk mencetak seluruh bab, kosongkan parameter (i.e. \bab{ })  
% Catatan: Jika ingin menambahkan bab ke-10 dan seterusnya, harus dilakukan 
% secara manual
%=============================================================================
\bab{5}
%=============================================================================

%_____________________________________________________________________________
%=============================================================================
% 								BAGIAN VII
%=============================================================================
% Lampiran yang akan dicetak: isi dengan huruf A,B,C s.d I, sehingga bisa 
% digunakan untuk mencetak hanya 1 atau beberapa lampiran saja
% Jika lebih dari 1 lampiran, pisahkan dengan ',', lampiran akan dicetak 
% terurut sesuai urutan lampiran (e.g. \bab{A,B,C}).
% Jika tidak ingin mencetak lampiran apapun, isi dengan -1 (i.e. \lampiran{-1})
% Untuk mencetak seluruh mapiran, kosongkan parameter (i.e. \lampiran{ })  
% Catatan: Jika ingin menambahkan lampiran ke-J dan seterusnya, harus 
% dilakukan secara manual
%=============================================================================
\lampiran{ }
%=============================================================================

%_____________________________________________________________________________
%=============================================================================
% 								BAGIAN VIII
%=============================================================================
% Data diri dan skripsi/tugas akhir
% - namanpm: Nama dan NPM anda, penggunaan huruf besar untuk nama harus benar
%			 dan gunakan 10 digit npm UNPAR, PASTIKAN BAHWA BENAR !!!
%			 (e.g. \namanpm{Jane Doe}{1992710001}
% - judul : Dalam bahasa Indonesia, perhatikan penggunaan huruf besar, judul
%			tidak menggunakan huruf besar seluruhnya !!! 
% - tanggal : isi dengan {tangga}{bulan}{tahun} dalam angka numerik, jangan 
%			  menuliskan kata (e.g. AGUSTUS) dalam isian bulan
%			  Tanggal ini adalah tanggal dimana anda akan melaksanakan sidang 
%			  ujian akhir skripsi/tugas akhir
% - pembimbing: isi dengan pembimbing anda, lihat daftar dosen di file dosen.tex
%				jika pembimbing hanya 1, kosongkan parameter kedua 
%				(e.g. \pembimbing{\JND}{  } ) , \JND adalah kode dosen
% - penguji : isi dengan para penguji anda, lihat daftar dosen di file dosen.tex
%				(e.g. \penguji{\JHD}{\JCD} ) , \JND dan \JCD adalah kode dosen
% !!Lihat singkatan pembimbing dan penguji anda di file dosen.tex
%=============================================================================
\namanpm{Mohamad Fahrizal Septrianto}{2013730017}	%hilangkan tanda << & >>
\tanggal{25}{5}{2016}			%hilangkan tanda << & >>
\pembimbing{\VSM}{} %hilangkan tanda << & >>    
\penguji{\TAB}{} 				%hilangkan tanda << & >>
%=============================================================================

%_____________________________________________________________________________
%=============================================================================
% 								BAGIAN IX
%=============================================================================
% Judul dan title : judul bhs indonesia dan inggris
% - judulINA: judul dalam bahasa indonesia
% - judulENG: title in english
% PERHATIAN: - langsung mulai setelah '{' awal, jangan mulai menulis di baris 
%			   bawahnya
%			 - Gunakan \texorpdfstring{\\}{} untuk pindah ke baris baru
%			 - Judul TIDAK ditulis dengan menggunakan huruf besar seluruhnya !!
%			 - Gunakan perintah \texorpdfstring{\\}{} untuk baris baru
%=============================================================================
\judulINA{Klasifikasi Big Data dengan Algoritma Naive Bayes pada Sistem Terdistribusi Hadoop}
\judulENG{Big Data Classification With Naive Bayes Algorithm On Hadoop Distributed System}
%_____________________________________________________________________________
%=============================================================================
% 								BAGIAN X
%=============================================================================
% Abstrak dan abstract : abstrak bhs indonesia dan inggris
% - abstrakINA: abstrak bahasa indonesia
% - abstrakENG: abstract in english
% PERHATIAN: langsung mulai setelah '{' awal, jangan mulai menulis di baris 
%			 bawahnya
%=============================================================================
\abstrakINA{<<Tuliskan abstrak anda di sini, dalam bahasa Indonesia>> \\\kant[1]}
\abstrakENG{<<Tuliskan abstrak anda di sini, dalam bahasa Inggris>> \\\kant[2]} 
%=============================================================================

%_____________________________________________________________________________
%=============================================================================
% 								BAGIAN XI
%=============================================================================
% Kata-kata kunci dan keywords : diletakkan di bawah abstrak (ina dan eng)
% - kunciINA: kata-kata kunci dalam bahasa indonesia
% - kunciENG: keywords in english
%=============================================================================
\kunciINA{<<Tuliskan di sini kata-kata kunci yang anda gunakan, dalam bahasa Indonesia>>}
\kunciENG{<<Tuliskan di sini kata-kata kunci yang anda gunakan, dalam bahasa Inggris>>}
%=============================================================================

%_____________________________________________________________________________
%=============================================================================
% 								BAGIAN XII
%=============================================================================
% Persembahan : kepada siapa anda mempersembahkan skripsi ini ...
%=============================================================================
\untuk{<<kepada siapa anda mempersembahkan skripsi ini\ldots?>>}
%=============================================================================

%_____________________________________________________________________________
%=============================================================================
% 								BAGIAN XIII
%=============================================================================
% Kata Pengantar: tempat anda menuliskan kata pengantar dan ucapan terima 
% kasih kepada yang telah membantu anda bla bla bla ....  
%=============================================================================
\prakata{\kant[3-4]}
%=============================================================================

%_____________________________________________________________________________
%=============================================================================
% 								BAGIAN XIV
%=============================================================================
% Tambahkan hyphen (pemenggalan kata) yang anda butuhkan di sini 
%=============================================================================
\hyphenation{ma-te-ma-ti-ka}
\hyphenation{fi-si-ka}
\hyphenation{tek-nik}
\hyphenation{in-for-ma-ti-ka}
%=============================================================================

%_____________________________________________________________________________
%=============================================================================
% 								BAGIAN XV
%=============================================================================
% Tambahkan perintah yang anda buat sendiri di sini 
%=============================================================================
\newcommand{\vtemplateauthor}{lionov}
\pgfplotsset{compat=newest}
\usetikzlibrary{patterns}

% START CUSTOM BY MOHAMAD FAHRIZAL S.

\definecolor{dkgreen}{rgb}{0,0.6,0}
\definecolor{gray}{rgb}{0.5,0.5,0.5}
\definecolor{mauve}{rgb}{0.58,0,0.82}
\lstset{frame=tb,
  language=Java,
  aboveskip=3mm,
  belowskip=3mm,
  showstringspaces=false,
  columns=flexible,
  basicstyle={\small\ttfamily},
  %numbers=none,
  numberstyle=\tiny\color{gray},
  keywordstyle=\color{blue},
  commentstyle=\color{dkgreen},
  stringstyle=\color{mauve},
  breaklines=true,
  breakatwhitespace=true,
  tabsize=3
}
\setlength{\headsep}{30pt}

\setcounter{secnumdepth}{3}
\setcounter{tocdepth}{3}

\titleformat{\paragraph}
{\normalfont\normalsize\bfseries}{\theparagraph}{1em}{}
\titlespacing*{\paragraph}
{0pt}{3.25ex plus 1ex minus .2ex}{1.5ex plus .2ex}

% END CUSTOM BY MOHAMAD FAHRIZAL S.

%=============================================================================

% Copyright \textcopyright [Lionov] [09-10-2016]. All rights reserved